% ============================================================================
% article_template.tex - Complete Article Template
% ============================================================================
% This template demonstrates the structure of a typical academic article,
% including commonly used packages, metadata, and sectioning.
%
% Compilation:
%   pdflatex article_template.tex
% ============================================================================

\documentclass[11pt,a4paper]{article}

% ============================================================================
% PACKAGE IMPORTS
% ============================================================================

% Essential packages for encoding and fonts
\usepackage[utf8]{inputenc}      % UTF-8 input encoding
\usepackage[T1]{fontenc}         % Font encoding
\usepackage{lmodern}             % Latin Modern font (clearer than default)

% Page layout and margins
\usepackage[margin=1in]{geometry}

% Language and hyphenation
\usepackage[english]{babel}

% Enhanced mathematical typesetting
\usepackage{amsmath, amssymb, amsthm}

% Graphics and figures
\usepackage{graphicx}
\usepackage{float}               % Better float positioning

% Tables
\usepackage{booktabs}            % Professional table formatting

% Colors
\usepackage{xcolor}

% Hyperlinks (should be loaded last)
\usepackage{hyperref}
\hypersetup{
    colorlinks=true,
    linkcolor=blue,
    citecolor=green,
    urlcolor=cyan,
    pdftitle={Article Template},
    pdfauthor={Your Name}
}

% ============================================================================
% CUSTOM COMMANDS AND SETTINGS
% ============================================================================

% Define a custom theorem environment
\newtheorem{theorem}{Theorem}[section]
\newtheorem{lemma}[theorem]{Lemma}
\newtheorem{corollary}[theorem]{Corollary}

% Custom command example
\newcommand{\R}{\mathbb{R}}      % Real numbers symbol

% ============================================================================
% DOCUMENT METADATA
% ============================================================================

\title{A Comprehensive Article Template for \LaTeX}

\author{
    John Doe\thanks{Department of Computer Science, Example University} \\
    \texttt{john.doe@example.edu}
    \and
    Jane Smith\thanks{Department of Mathematics, Example University} \\
    \texttt{jane.smith@example.edu}
}

\date{\today}                    % Or specify a date: \date{January 1, 2026}

% ============================================================================
% DOCUMENT BODY
% ============================================================================

\begin{document}

% Generate title
\maketitle

% ============================================================================
% ABSTRACT
% ============================================================================

\begin{abstract}
This document serves as a comprehensive template for writing academic articles
in \LaTeX. It demonstrates the proper structure of an article, including the
use of sections, subsections, mathematical equations, figures, tables, and
references. The abstract should provide a concise summary of the paper's
purpose, methods, results, and conclusions, typically in 150--250 words.
This template includes commonly used packages and best practices for
scientific writing.

\noindent\textbf{Keywords:} \LaTeX, document structure, academic writing,
template, typesetting
\end{abstract}

% ============================================================================
% MAIN CONTENT
% ============================================================================

\section{Introduction}
\label{sec:introduction}

The introduction sets the context for your work and states the problem you
are addressing. It should provide sufficient background for readers to
understand the motivation and significance of your research.

\LaTeX{} is a high-quality typesetting system designed for the production of
technical and scientific documentation. Unlike word processors, \LaTeX{}
separates content from formatting, allowing authors to focus on writing while
the system handles layout and typography.

This template demonstrates:
\begin{itemize}
    \item Proper document structure with sections and subsections
    \item Mathematical equation formatting
    \item Figure and table inclusion
    \item Cross-referencing and citations
    \item Bibliography management
\end{itemize}

\section{Background and Related Work}
\label{sec:background}

This section reviews relevant literature and establishes the theoretical
foundation for your work. You can cite references using \verb|\cite{}|
commands (requires a bibliography).

\subsection{Document Classes}

\LaTeX{} provides several standard document classes:
\begin{description}
    \item[article] For short documents, journal papers, conference proceedings
    \item[report] For longer documents with chapters (theses, books)
    \item[book] For actual books with front/back matter
    \item[beamer] For presentations
\end{description}

\subsection{Package Ecosystem}

The Comprehensive \TeX{} Archive Network (CTAN) hosts thousands of packages
that extend \LaTeX's capabilities. Common packages include \texttt{amsmath}
for mathematics, \texttt{graphicx} for figures, and \texttt{hyperref} for
hyperlinks.

\section{Methodology}
\label{sec:methodology}

Describe your methods here. You can cross-reference other sections using
\verb|\label{}| and \verb|\ref{}| commands. For example, the introduction
is in Section~\ref{sec:introduction}.

\subsection{Mathematical Notation}

Inline mathematics uses single dollar signs: $E = mc^2$. Display mathematics
uses the \texttt{equation} environment:

\begin{equation}
    \label{eq:quadratic}
    x = \frac{-b \pm \sqrt{b^2 - 4ac}}{2a}
\end{equation}

We can reference Equation~\ref{eq:quadratic} later in the text.

\subsection{Lists and Enumerations}

Unordered lists (itemize):
\begin{itemize}
    \item First item
    \item Second item
    \item Third item
\end{itemize}

Ordered lists (enumerate):
\begin{enumerate}
    \item First step
    \item Second step
    \item Third step
\end{enumerate}

\section{Results}
\label{sec:results}

Present your findings in this section. Use subsections to organize different
aspects of your results.

\subsection{Numerical Results}

Table~\ref{tab:results} shows example numerical data.

\begin{table}[h]
\centering
\caption{Example results table}
\label{tab:results}
\begin{tabular}{lrrr}
\toprule
Method & Accuracy & Precision & Recall \\
\midrule
Method A & 0.85 & 0.82 & 0.88 \\
Method B & 0.92 & 0.90 & 0.94 \\
Method C & 0.88 & 0.85 & 0.91 \\
\bottomrule
\end{tabular}
\end{table}

\subsection{Theoretical Results}

\begin{theorem}[Pythagorean Theorem]
\label{thm:pythagoras}
In a right triangle, the square of the hypotenuse equals the sum of squares
of the other two sides:
\[
    a^2 + b^2 = c^2
\]
\end{theorem}

\section{Discussion}
\label{sec:discussion}

Interpret your results and discuss their implications. Compare with existing
work and acknowledge limitations.

\section{Conclusion}
\label{sec:conclusion}

Summarize your main contributions and suggest directions for future work.
This template provides a solid foundation for academic article writing in
\LaTeX, demonstrating best practices for document organization and formatting.

% ============================================================================
% ACKNOWLEDGMENTS
% ============================================================================

\section*{Acknowledgments}

This work was supported by Example Grant No. 12345. The authors thank
Dr. Example Reviewer for helpful comments.

% ============================================================================
% BIBLIOGRAPHY
% ============================================================================
% In a real document, you would use BibTeX:
% \bibliographystyle{plain}
% \bibliography{references}
%
% For this template, we show a manual bibliography:

\begin{thebibliography}{9}

\bibitem{lamport1994}
Leslie Lamport.
\textit{\LaTeX: A Document Preparation System}.
Addison-Wesley, 2nd edition, 1994.

\bibitem{knuth1984}
Donald E. Knuth.
\textit{The \TeX{}book}.
Addison-Wesley, 1984.

\bibitem{mittelbach2004}
Frank Mittelbach and Michel Goossens.
\textit{The \LaTeX{} Companion}.
Addison-Wesley, 2nd edition, 2004.

\end{thebibliography}

\end{document}

% ============================================================================
% COMPILATION NOTES
% ============================================================================
% For documents with bibliography and cross-references:
%   1. pdflatex article_template.tex
%   2. bibtex article_template (if using BibTeX)
%   3. pdflatex article_template.tex
%   4. pdflatex article_template.tex
%
% Modern alternative: use latexmk for automatic compilation
%   latexmk -pdf article_template.tex
% ============================================================================
