% ============================================================================
% report_template.tex - Complete Report/Thesis Template
% ============================================================================
% This template demonstrates the structure of a longer document using the
% report class, including chapters, table of contents, list of figures,
% and list of tables.
%
% Compilation:
%   pdflatex report_template.tex (run twice for TOC)
% ============================================================================

\documentclass[12pt,a4paper,oneside]{report}

% ============================================================================
% PACKAGE IMPORTS
% ============================================================================

\usepackage[utf8]{inputenc}
\usepackage[T1]{fontenc}
\usepackage{lmodern}

% Page layout
\usepackage[top=1in, bottom=1in, left=1.25in, right=1in]{geometry}

% Language
\usepackage[english]{babel}

% Mathematics
\usepackage{amsmath, amssymb, amsthm}

% Graphics
\usepackage{graphicx}
\usepackage{float}

% Tables
\usepackage{booktabs}
\usepackage{multirow}

% Colors
\usepackage{xcolor}

% Code listings
\usepackage{listings}
\lstset{
    basicstyle=\ttfamily\small,
    breaklines=true,
    frame=single,
    numbers=left,
    numberstyle=\tiny,
    commentstyle=\color{gray},
    keywordstyle=\color{blue},
    stringstyle=\color{red}
}

% Better captions
\usepackage{caption}
\usepackage{subcaption}

% Header and footer customization
\usepackage{fancyhdr}
\pagestyle{fancy}
\fancyhf{}
\fancyhead[L]{\leftmark}
\fancyhead[R]{\thepage}
\renewcommand{\headrulewidth}{0.4pt}

% Hyperlinks (load last)
\usepackage{hyperref}
\hypersetup{
    colorlinks=true,
    linkcolor=blue,
    citecolor=green,
    urlcolor=cyan,
    pdftitle={Report Template},
    pdfauthor={Your Name}
}

% ============================================================================
% THEOREM ENVIRONMENTS
% ============================================================================

\newtheorem{theorem}{Theorem}[chapter]
\newtheorem{lemma}[theorem]{Lemma}
\newtheorem{proposition}[theorem]{Proposition}
\newtheorem{corollary}[theorem]{Corollary}

\theoremstyle{definition}
\newtheorem{definition}[theorem]{Definition}
\newtheorem{example}[theorem]{Example}

\theoremstyle{remark}
\newtheorem{remark}[theorem]{Remark}

% ============================================================================
% CUSTOM COMMANDS
% ============================================================================

\newcommand{\R}{\mathbb{R}}
\newcommand{\N}{\mathbb{N}}
\newcommand{\Z}{\mathbb{Z}}
\newcommand{\Q}{\mathbb{Q}}
\newcommand{\C}{\mathbb{C}}

% ============================================================================
% TITLE PAGE INFORMATION
% ============================================================================

\title{
    {\Huge\textbf{A Comprehensive Guide to \LaTeX{} Report Writing}} \\
    \vspace{0.5cm}
    {\Large Demonstrating Document Structure and Best Practices}
}

\author{
    Jane Smith \\
    Student ID: 123456789 \\
    \vspace{0.5cm}
    Department of Computer Science \\
    Example University \\
    \vspace{0.5cm}
    \texttt{jane.smith@example.edu}
}

\date{January 2026}

% ============================================================================
% DOCUMENT BODY
% ============================================================================

\begin{document}

% ============================================================================
% FRONT MATTER
% ============================================================================

% Custom title page
\begin{titlepage}
    \centering
    \vspace*{2cm}

    {\Huge\textbf{A Comprehensive Guide to}} \\
    \vspace{0.5cm}
    {\Huge\textbf{\LaTeX{} Report Writing}} \\
    \vspace{1cm}
    {\Large Demonstrating Document Structure and Best Practices} \\

    \vfill

    {\Large\textbf{Jane Smith}} \\
    \vspace{0.3cm}
    Student ID: 123456789 \\

    \vfill

    A report submitted in partial fulfillment \\
    of the requirements for the degree of \\
    \vspace{0.3cm}
    {\large Bachelor of Science} \\
    \vspace{0.3cm}
    in \\
    \vspace{0.3cm}
    {\large Computer Science} \\

    \vfill

    {\large Department of Computer Science} \\
    {\large Example University} \\

    \vfill

    {\large January 2026}

\end{titlepage}

% Roman numerals for front matter
\pagenumbering{roman}

% ============================================================================
% ABSTRACT
% ============================================================================

\chapter*{Abstract}
\addcontentsline{toc}{chapter}{Abstract}

This report provides a comprehensive template for writing technical reports,
theses, and dissertations using the \LaTeX{} document preparation system.
It demonstrates the use of the \texttt{report} document class, which is
designed for longer documents that are organized into chapters rather than
sections.

The template includes examples of front matter (abstract, acknowledgments,
table of contents), main content organized into chapters with sections and
subsections, mathematical notation, figures, tables, code listings, and
bibliography. It follows best practices for academic writing and document
structure.

Key features demonstrated include automatic table of contents generation,
cross-referencing, theorem environments, and professional formatting suitable
for academic submission. This template can be easily adapted for various
types of technical reports and academic documents.

% ============================================================================
% ACKNOWLEDGMENTS
% ============================================================================

\chapter*{Acknowledgments}
\addcontentsline{toc}{chapter}{Acknowledgments}

I would like to express my sincere gratitude to my supervisor, Dr. John Doe,
for his invaluable guidance and support throughout this project. His expertise
and insights have been instrumental in shaping this work.

I am also grateful to my colleagues in the Computer Science Department for
their helpful discussions and feedback. Special thanks to the \LaTeX{}
community for creating and maintaining this excellent typesetting system.

Finally, I thank my family and friends for their constant encouragement and
support.

% ============================================================================
% TABLE OF CONTENTS
% ============================================================================

\tableofcontents

\listoffigures
\addcontentsline{toc}{chapter}{List of Figures}

\listoftables
\addcontentsline{toc}{chapter}{List of Tables}

% ============================================================================
% MAIN MATTER
% ============================================================================

% Clear page and start Arabic numbering
\cleardoublepage
\pagenumbering{arabic}

% ============================================================================
% CHAPTER 1: INTRODUCTION
% ============================================================================

\chapter{Introduction}
\label{ch:introduction}

This chapter introduces the purpose and scope of the report, provides
background information, and outlines the structure of the document.

\section{Motivation}
\label{sec:motivation}

\LaTeX{} is the de facto standard for scientific and technical document
preparation. Unlike word processors, \LaTeX{} provides:

\begin{itemize}
    \item Professional typesetting quality
    \item Consistent formatting throughout the document
    \item Excellent mathematical equation support
    \item Automatic numbering and cross-referencing
    \item Efficient handling of bibliography
    \item Platform independence
\end{itemize}

\section{Objectives}

The primary objectives of this template are to:

\begin{enumerate}
    \item Demonstrate the structure of a \texttt{report} class document
    \item Showcase best practices for academic writing in \LaTeX
    \item Provide reusable examples for common document elements
    \item Serve as a starting point for thesis and report writing
\end{enumerate}

\section{Document Structure}

This report is organized as follows:

\begin{description}
    \item[Chapter~\ref{ch:introduction}] provides the introduction and motivation
    \item[Chapter~\ref{ch:literature}] reviews relevant literature and background
    \item[Chapter~\ref{ch:methodology}] describes the methodology and approach
    \item[Chapter~\ref{ch:results}] presents the results and findings
    \item[Chapter~\ref{ch:conclusion}] concludes and suggests future work
\end{description}

% ============================================================================
% CHAPTER 2: LITERATURE REVIEW
% ============================================================================

\chapter{Literature Review}
\label{ch:literature}

This chapter reviews relevant literature and establishes theoretical
foundations. In a real report, you would cite published works extensively.

\section{History of \TeX{} and \LaTeX}

\TeX{} was created by Donald Knuth in 1978 as a typesetting system designed
to handle complex mathematical formulas. \LaTeX{}, developed by Leslie Lamport
in 1984, is a macro package built on top of \TeX{} that simplifies document
preparation.

\section{Document Classes}

The choice of document class affects the overall structure and available
commands. Key differences between \texttt{article} and \texttt{report}:

\begin{table}[h]
\centering
\caption{Comparison of article and report classes}
\label{tab:classes}
\begin{tabular}{lll}
\toprule
Feature & Article & Report \\
\midrule
Top-level structure & \verb|\section| & \verb|\chapter| \\
Page numbering & Continuous & Per chapter option \\
Typical length & Short (journal papers) & Long (theses, books) \\
Title page & On first page & Separate page \\
Chapter breaks & N/A & New page \\
\bottomrule
\end{tabular}
\end{table}

\section{Mathematical Typesetting}

One of \LaTeX's greatest strengths is mathematical typesetting. Consider
the famous equation:

\begin{equation}
\label{eq:euler}
e^{i\pi} + 1 = 0
\end{equation}

Equation~\ref{eq:euler}, known as Euler's identity, elegantly combines five
fundamental mathematical constants.

% ============================================================================
% CHAPTER 3: METHODOLOGY
% ============================================================================

\chapter{Methodology}
\label{ch:methodology}

This chapter describes the methods and techniques used. In a technical report,
you would provide detailed descriptions of your approach.

\section{Experimental Setup}

\begin{definition}[Sample Space]
A sample space $\Omega$ is the set of all possible outcomes of a random
experiment.
\end{definition}

\begin{theorem}[Law of Large Numbers]
\label{thm:lln}
Let $X_1, X_2, \ldots$ be a sequence of independent and identically
distributed random variables with mean $\mu$. Then:
\[
\lim_{n \to \infty} \frac{1}{n} \sum_{i=1}^n X_i = \mu
\]
almost surely.
\end{theorem}

\section{Implementation}

Code listings can be included using the \texttt{listings} package:

\begin{lstlisting}[language=Python, caption=Example Python code, label=lst:python]
def fibonacci(n):
    """Calculate the nth Fibonacci number."""
    if n <= 1:
        return n
    return fibonacci(n-1) + fibonacci(n-2)

# Example usage
for i in range(10):
    print(f"F({i}) = {fibonacci(i)}")
\end{lstlisting}

Listing~\ref{lst:python} shows a simple recursive implementation of the
Fibonacci sequence.

% ============================================================================
% CHAPTER 4: RESULTS
% ============================================================================

\chapter{Results and Discussion}
\label{ch:results}

This chapter presents the results of the work and discusses their implications.

\section{Quantitative Results}

Table~\ref{tab:performance} summarizes the performance metrics.

\begin{table}[h]
\centering
\caption{Performance comparison of different methods}
\label{tab:performance}
\begin{tabular}{lrrr}
\toprule
Method & Accuracy (\%) & Time (ms) & Memory (MB) \\
\midrule
Method A & 85.3 & 120 & 256 \\
Method B & 92.1 & 95 & 384 \\
Method C & 88.7 & 110 & 320 \\
\textbf{Proposed} & \textbf{94.5} & \textbf{85} & \textbf{280} \\
\bottomrule
\end{tabular}
\end{table}

\section{Analysis}

The proposed method achieves the best performance across all metrics,
demonstrating a 2.4\% improvement in accuracy while reducing computation
time by 10.5\% compared to the second-best method.

% ============================================================================
% CHAPTER 5: CONCLUSION
% ============================================================================

\chapter{Conclusion and Future Work}
\label{ch:conclusion}

\section{Summary}

This report has demonstrated the structure and capabilities of the \LaTeX{}
\texttt{report} document class. Key features include:

\begin{itemize}
    \item Chapter-based organization
    \item Automatic front matter generation
    \item Professional formatting and typography
    \item Comprehensive cross-referencing system
\end{itemize}

\section{Future Work}

Future enhancements to this template could include:

\begin{enumerate}
    \item Integration with bibliography management tools (BibTeX, biblatex)
    \item Custom chapter and section heading styles
    \item Advanced figure and table formatting
    \item Integration with version control systems
\end{enumerate}

% ============================================================================
% BIBLIOGRAPHY
% ============================================================================

\begin{thebibliography}{99}
\addcontentsline{toc}{chapter}{Bibliography}

\bibitem{lamport1994}
Leslie Lamport.
\textit{\LaTeX: A Document Preparation System}.
Addison-Wesley Professional, 2nd edition, 1994.

\bibitem{knuth1984}
Donald E. Knuth.
\textit{The \TeX{}book}.
Addison-Wesley, 1984.

\bibitem{mittelbach2004}
Frank Mittelbach and Michel Goossens.
\textit{The \LaTeX{} Companion}.
Addison-Wesley Professional, 2nd edition, 2004.

\bibitem{kottwitz2011}
Stefan Kottwitz.
\textit{\LaTeX{} Beginner's Guide}.
Packt Publishing, 2011.

\end{thebibliography}

% ============================================================================
% APPENDICES
% ============================================================================

\appendix

\chapter{Additional Information}

Appendices contain supplementary material that supports the main text but
would interrupt the flow if included in the main chapters.

\section{Installation Instructions}

Instructions for installing \LaTeX{} on various platforms would go here.

\section{Complete Code Listings}

Full source code and additional examples would be included here.

\end{document}
