% ============================================================================
% equation_numbering.tex - Equation Numbering and Cross-Referencing
% ============================================================================
% This document demonstrates how to number equations and create cross-
% references in LaTeX using various math environments.
%
% Compilation:
%   pdflatex equation_numbering.tex (run twice for references)
% ============================================================================

\documentclass[11pt]{article}

\usepackage{amsmath}    % Essential for advanced math environments
\usepackage{amssymb}
\usepackage{amsthm}

\usepackage[margin=1in]{geometry}
\usepackage[T1]{fontenc}
\usepackage{lmodern}

% Hyperlinks with cross-references
\usepackage{hyperref}
\hypersetup{
    colorlinks=true,
    linkcolor=blue,
    citecolor=green,
    urlcolor=cyan
}

% For colored boxes
\usepackage{xcolor}
\usepackage{tcolorbox}

\title{Equation Numbering and Cross-Referencing in \LaTeX}
\author{Mathematics Department}
\date{\today}

\begin{document}

\maketitle

\tableofcontents

\section{Introduction}

This document demonstrates how to number mathematical equations and create
cross-references to them. Proper equation numbering and referencing is
essential for academic writing, allowing readers to easily locate and
reference specific formulas.

% ============================================================================
\section{The \texttt{equation} Environment}
% ============================================================================

The \texttt{equation} environment creates a single numbered equation centered
on its own line.

\subsection{Basic Usage}

The most fundamental quadratic equation is:

\begin{equation}
\label{eq:quadratic}
x = \frac{-b \pm \sqrt{b^2 - 4ac}}{2a}
\end{equation}

We can reference Equation~\ref{eq:quadratic} using \verb|\ref{eq:quadratic}|.
With hyperref, clicking the reference number jumps to the equation.

\subsection{The \texttt{\textbackslash label} and \texttt{\textbackslash ref} Commands}

Every numbered equation can have a label defined with \verb|\label{...}|:

\begin{equation}
\label{eq:euler}
e^{i\pi} + 1 = 0
\end{equation}

Euler's identity (Equation~\ref{eq:euler}) is often called the most beautiful
equation in mathematics.

\begin{tcolorbox}[colback=yellow!10, colframe=orange!80, title=Best Practice]
\textbf{Label naming convention:} Use descriptive labels with prefixes like
\texttt{eq:}, \texttt{fig:}, \texttt{tab:}, \texttt{sec:} to identify the
type of element being referenced.
\end{tcolorbox}

\subsection{The \texttt{\textbackslash eqref} Command}

The \verb|\eqref| command automatically adds parentheses around equation numbers:

\begin{equation}
\label{eq:pythagoras}
a^2 + b^2 = c^2
\end{equation}

Compare \verb|\ref{eq:pythagoras}| which gives \ref{eq:pythagoras} versus
\verb|\eqref{eq:pythagoras}| which gives \eqref{eq:pythagoras}. The latter
is preferred for equation references.

% ============================================================================
\section{Unnumbered Equations}
% ============================================================================

Sometimes you want display mathematics without a number.

\subsection{Using \textbackslash[\ldots\textbackslash]}

The simplest way is to use \verb|\[...\]| or \verb|$$...$$| (though the former
is preferred in LaTeX):

\[
\int_0^\infty e^{-x^2} dx = \frac{\sqrt{\pi}}{2}
\]

\subsection{Using \texttt{equation*}}

Alternatively, use the \texttt{equation*} environment (requires \texttt{amsmath}):

\begin{equation*}
\sum_{n=1}^\infty \frac{1}{n^2} = \frac{\pi^2}{6}
\end{equation*}

This is useful when you want to maintain consistency in style, changing only
the environment name to toggle numbering.

% ============================================================================
\section{The \texttt{align} Environment}
% ============================================================================

The \texttt{align} environment is used for multi-line equations that should
be aligned, typically at an equals sign or relation symbol.

\subsection{Basic Alignment}

Use \verb|&| to specify alignment points and \verb|\\| for line breaks:

\begin{align}
\label{eq:align1}
f(x) &= x^2 + 2x + 1 \\
\label{eq:align2}
     &= (x + 1)^2 \\
\label{eq:align3}
     &= (x + 1)(x + 1)
\end{align}

Each line gets its own number. We can reference individual lines:
Equation~\eqref{eq:align1} shows the expanded form,
Equation~\eqref{eq:align2} shows the factored form.

\subsection{Suppressing Numbering on Specific Lines}

Use \verb|\notag| to suppress numbering on individual lines:

\begin{align}
x^2 + y^2 &= r^2 \label{eq:circle} \\
\frac{x^2}{a^2} + \frac{y^2}{b^2} &= 1 \notag \\
y &= mx + b \label{eq:line}
\end{align}

Only Equations~\eqref{eq:circle} and \eqref{eq:line} are numbered.

\subsection{Multiple Alignments}

You can have multiple alignment points by using additional \verb|&| symbols:

\begin{align}
x &= y & X &= Y & a &= b+c \\
x' &= y' & X' &= Y' & a' &= b
\end{align}

Pairs of \verb|&| symbols: first for alignment, second to separate equations.

\subsection{Unnumbered Version: \texttt{align*}}

The \texttt{align*} environment produces no equation numbers:

\begin{align*}
\sin^2 x + \cos^2 x &= 1 \\
1 + \tan^2 x &= \sec^2 x \\
1 + \cot^2 x &= \csc^2 x
\end{align*}

% ============================================================================
\section{The \texttt{gather} Environment}
% ============================================================================

The \texttt{gather} environment is for multiple equations that should be
centered but not aligned:

\begin{gather}
\label{eq:gather1}
a = b + c \\
\label{eq:gather2}
x^2 + y^2 = z^2 \\
\label{eq:gather3}
E = mc^2
\end{gather}

Each equation is centered independently and numbered. Reference example:
Equation~\eqref{eq:gather2} is the Pythagorean theorem.

Unnumbered version (\texttt{gather*}):

\begin{gather*}
\nabla \cdot \mathbf{E} = \frac{\rho}{\epsilon_0} \\
\nabla \cdot \mathbf{B} = 0 \\
\nabla \times \mathbf{E} = -\frac{\partial \mathbf{B}}{\partial t} \\
\nabla \times \mathbf{B} = \mu_0 \mathbf{J} + \mu_0 \epsilon_0
\frac{\partial \mathbf{E}}{\partial t}
\end{gather*}

% ============================================================================
\section{The \texttt{multline} Environment}
% ============================================================================

The \texttt{multline} environment is for single equations that are too long
for one line:

\begin{multline}
\label{eq:multline}
p(x) = 3x^6 + 14x^5y + 590x^4y^2 + 19x^3y^3 \\
- 12x^2y^4 - 12xy^5 + 2y^6 - a^3b^3
\end{multline}

The first line is left-aligned, the last line is right-aligned, and any
intermediate lines are centered. The entire equation gets one number.

Reference: The polynomial in Equation~\eqref{eq:multline} has six terms.

Unnumbered version (\texttt{multline*}):

\begin{multline*}
\int_0^1 \int_0^1 \int_0^1 (x^2 + y^2 + z^2)^{10} \, dx \, dy \, dz \\
= \text{(some very complicated expression)}
\end{multline*}

% ============================================================================
\section{The \texttt{split} Environment}
% ============================================================================

The \texttt{split} environment is used within another math environment (like
\texttt{equation}) to split a single equation across multiple lines with
alignment:

\begin{equation}
\label{eq:split}
\begin{split}
(x + y)^3 &= (x + y)(x + y)^2 \\
          &= (x + y)(x^2 + 2xy + y^2) \\
          &= x^3 + 3x^2y + 3xy^2 + y^3
\end{split}
\end{equation}

The entire \texttt{split} environment gets one equation number. This is
useful when you want one number for a multi-line derivation.

Reference: The binomial expansion in Equation~\eqref{eq:split} shows the
cube of a sum.

% ============================================================================
\section{Subequations}
% ============================================================================

The \texttt{subequations} environment numbers equations with sub-labels
(a, b, c, etc.):

\begin{subequations}
\label{eq:maxwell}
Maxwell's equations in differential form:
\begin{align}
\nabla \cdot \mathbf{E} &= \frac{\rho}{\epsilon_0} \label{eq:maxwell_gauss} \\
\nabla \cdot \mathbf{B} &= 0 \label{eq:maxwell_nomonopole} \\
\nabla \times \mathbf{E} &= -\frac{\partial \mathbf{B}}{\partial t}
\label{eq:maxwell_faraday} \\
\nabla \times \mathbf{B} &= \mu_0 \mathbf{J} +
\mu_0\epsilon_0\frac{\partial \mathbf{E}}{\partial t}
\label{eq:maxwell_ampere}
\end{align}
\end{subequations}

We can reference the entire group as Equations~\eqref{eq:maxwell}, or
individual equations: Gauss's law is \eqref{eq:maxwell_gauss}, Faraday's
law is \eqref{eq:maxwell_faraday}.

% ============================================================================
\section{The \texttt{cases} Environment}
% ============================================================================

The \texttt{cases} environment is for piecewise functions:

\begin{equation}
\label{eq:absolute}
|x| = \begin{cases}
x & \text{if } x \geq 0 \\
-x & \text{if } x < 0
\end{cases}
\end{equation}

The absolute value function is defined in Equation~\eqref{eq:absolute}.

Another example:

\begin{equation}
\label{eq:piecewise}
f(x) = \begin{cases}
0 & \text{if } x < 0 \\
x^2 & \text{if } 0 \leq x < 1 \\
2 - x & \text{if } 1 \leq x < 2 \\
0 & \text{if } x \geq 2
\end{cases}
\end{equation}

% ============================================================================
\section{Custom Numbering and Tags}
% ============================================================================

\subsection{Manual Tags}

You can override automatic numbering with \verb|\tag|:

\begin{equation}
E = mc^2 \tag{Einstein}
\end{equation}

\begin{equation}
a^2 + b^2 = c^2 \tag{Pythagoras}
\end{equation}

For starred tags without parentheses:

\begin{equation}
\int_a^b f(x) \, dx \tag*{Fundamental Theorem}
\end{equation}

\subsection{Equation Numbering by Section}

To number equations by section (e.g., 2.1, 2.2, etc.), add this to the preamble:

\begin{verbatim}
\numberwithin{equation}{section}
\end{verbatim}

% Example (commented out to not affect this document):
% \numberwithin{equation}{section}

% ============================================================================
\section{Cross-Referencing Best Practices}
% ============================================================================

\subsection{Label Placement}

Place \verb|\label| immediately after the item you want to reference:

\begin{itemize}
    \item For equations: right after \verb|\begin{equation}| or on the line
    \item For sections: right after \verb|\section{...}|
    \item For figures/tables: inside the caption
\end{itemize}

\subsection{Descriptive Labels}

Use meaningful labels:

\begin{itemize}
    \item \textcolor{green}{\checkmark} \verb|eq:mass_energy_equivalence|
    \item \textcolor{red}{$\times$} \verb|eq:eq1|
\end{itemize}

\subsection{The \texttt{cleveref} Package}

For even more sophisticated referencing, consider the \texttt{cleveref} package
(not loaded in this document):

\begin{verbatim}
\usepackage{cleveref}
% Then you can use:
\cref{eq:quadratic}     % produces "equation (1)"
\Cref{eq:quadratic}     % produces "Equation (1)"
\end{verbatim}

% ============================================================================
\section{Summary Example}
% ============================================================================

Let's combine several concepts. Consider the Taylor series expansion:

\begin{subequations}
\label{eq:taylor_series}
\begin{align}
f(x) &= \sum_{n=0}^\infty \frac{f^{(n)}(a)}{n!}(x-a)^n \label{eq:taylor_general} \\
e^x &= \sum_{n=0}^\infty \frac{x^n}{n!} = 1 + x + \frac{x^2}{2!} +
\frac{x^3}{3!} + \cdots \label{eq:taylor_exp} \\
\sin x &= \sum_{n=0}^\infty \frac{(-1)^n}{(2n+1)!}x^{2n+1} = x - \frac{x^3}{3!}
+ \frac{x^5}{5!} - \cdots \label{eq:taylor_sin} \\
\cos x &= \sum_{n=0}^\infty \frac{(-1)^n}{(2n)!}x^{2n} = 1 - \frac{x^2}{2!} +
\frac{x^4}{4!} - \cdots \label{eq:taylor_cos}
\end{align}
\end{subequations}

The Taylor series (Equations~\ref{eq:taylor_series}) provide polynomial
approximations to functions. The general form is given by
\eqref{eq:taylor_general}, with specific examples for the exponential function
\eqref{eq:taylor_exp}, sine \eqref{eq:taylor_sin}, and cosine
\eqref{eq:taylor_cos}.

From these, we can derive Euler's formula:

\begin{equation}
\label{eq:euler_formula}
e^{ix} = \cos x + i \sin x
\end{equation}

Substituting $x = \pi$ into Equation~\eqref{eq:euler_formula} yields Euler's
identity (Equation~\ref{eq:euler} from Section~2.2).

% ============================================================================
\section{Conclusion}
% ============================================================================

This document has demonstrated the major methods for numbering and
referencing equations in \LaTeX:

\begin{itemize}
    \item \texttt{equation}: single numbered equation
    \item \texttt{align}: aligned multi-line equations
    \item \texttt{gather}: centered multi-line equations
    \item \texttt{multline}: single equation split across lines
    \item \texttt{split}: aligned equation with single number
    \item \texttt{subequations}: grouped equations with sub-numbering
    \item \texttt{cases}: piecewise functions
\end{itemize}

Remember:
\begin{enumerate}
    \item Use \verb|\label| to mark equations
    \item Use \verb|\ref| or \verb|\eqref| to reference them
    \item Compile twice for references to resolve
    \item Use descriptive label names with prefixes
\end{enumerate}

\end{document}

% ============================================================================
% COMPILATION TIP
% ============================================================================
% You MUST compile TWICE for cross-references to work:
%   pdflatex equation_numbering.tex
%   pdflatex equation_numbering.tex
%
% On the first pass, LaTeX collects label information.
% On the second pass, it resolves the references.
% ============================================================================
