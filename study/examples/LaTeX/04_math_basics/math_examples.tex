% ============================================================================
% math_examples.tex - Basic Mathematical Typesetting in LaTeX
% ============================================================================
% This document demonstrates fundamental mathematical typesetting features
% in LaTeX, including inline and display mathematics, fractions, exponents,
% Greek letters, and common mathematical operators.
%
% Compilation:
%   pdflatex math_examples.tex
% ============================================================================

\documentclass[11pt]{article}

% Essential math packages
\usepackage{amsmath}    % Enhanced math environments
\usepackage{amssymb}    % Additional math symbols
\usepackage{amsfonts}   % Mathematical fonts

% Page layout
\usepackage[margin=1in]{geometry}

% Better font encoding
\usepackage[T1]{fontenc}
\usepackage{lmodern}

% Color for highlighting
\usepackage{xcolor}

% Hyperlinks
\usepackage{hyperref}
\hypersetup{colorlinks=true, linkcolor=blue}

\title{Basic Mathematical Typesetting in \LaTeX}
\author{Mathematics Department}
\date{\today}

\begin{document}

\maketitle

\tableofcontents

\section{Introduction}

\LaTeX{} excels at typesetting mathematical content. This document demonstrates
basic mathematical features that every \LaTeX{} user should know.

% ============================================================================
\section{Inline vs. Display Mathematics}
% ============================================================================

\subsection{Inline Mathematics}

Inline mathematics is embedded within text using single dollar signs or
\verb|\(...\)|. For example, the quadratic formula is $x = \frac{-b \pm
\sqrt{b^2-4ac}}{2a}$ and the famous equation $E=mc^2$ shows mass-energy
equivalence.

The Pythagorean theorem states that $a^2 + b^2 = c^2$ for a right triangle.
We can also write inequalities like $0 < x \leq 10$ or $-\infty < y < \infty$.

\subsection{Display Mathematics}

Display mathematics is set apart from text, centered and on its own line.
Use double dollar signs \verb|$$...$$| or \verb|\[...\]| for unnumbered
equations:

\[
\int_0^\infty e^{-x^2} \, dx = \frac{\sqrt{\pi}}{2}
\]

For numbered equations that can be referenced, use the \texttt{equation}
environment (see Section~\ref{sec:numbering} for details).

% ============================================================================
\section{Fractions and Binomial Coefficients}
% ============================================================================

\subsection{Fractions}

The \verb|\frac{numerator}{denominator}| command creates fractions:

\[
\frac{a}{b}, \quad \frac{x^2 + y^2}{z^2}, \quad
\frac{1}{1 + \frac{1}{1 + \frac{1}{x}}}
\]

In inline mode, fractions appear smaller: $\frac{1}{2}$ or $\frac{a+b}{c+d}$.
For better inline appearance, use \verb|\tfrac| (text-style fraction) or
\verb|/|: compare $\frac{1}{2}$ vs. $\tfrac{1}{2}$ vs. $1/2$.

For display-style fractions in text, use \verb|\dfrac|: $\dfrac{a}{b}$.

\subsection{Binomial Coefficients}

Binomial coefficients use \verb|\binom{n}{k}|:

\[
\binom{n}{k} = \frac{n!}{k!(n-k)!}, \quad
\binom{5}{2} = 10, \quad
\sum_{k=0}^n \binom{n}{k} = 2^n
\]

% ============================================================================
\section{Exponents and Subscripts}
% ============================================================================

\subsection{Superscripts (Exponents)}

Use the caret \verb|^| for superscripts:

\[
x^2, \quad y^{10}, \quad e^{i\pi}, \quad 2^{2^{2^2}} = 2^{16} = 65536
\]

For multiple characters in the exponent, use braces: $x^{n+1}$ not $x^n+1$.

\subsection{Subscripts}

Use underscore \verb|_| for subscripts:

\[
x_1, \quad a_{ij}, \quad \sum_{i=1}^n x_i, \quad
\lim_{x \to \infty} f(x)
\]

\subsection{Combining Subscripts and Superscripts}

You can combine both (order doesn't matter):

\[
x_i^2, \quad x^2_i, \quad \sum_{i=1}^{n} x_i^2, \quad
{}_nP_r = \frac{n!}{(n-r)!}
\]

% ============================================================================
\section{Greek Letters}
% ============================================================================

\subsection{Lowercase Greek Letters}

Greek letters are fundamental in mathematics. Use backslash followed by the
letter name:

\[
\alpha, \beta, \gamma, \delta, \epsilon, \varepsilon, \zeta, \eta, \theta,
\vartheta, \iota, \kappa, \lambda, \mu, \nu, \xi, \pi, \varpi, \rho, \varrho,
\sigma, \varsigma, \tau, \upsilon, \phi, \varphi, \chi, \psi, \omega
\]

Note the variant forms: $\epsilon$ vs. $\varepsilon$, $\theta$ vs. $\vartheta$,
$\phi$ vs. $\varphi$, $\pi$ vs. $\varpi$, $\rho$ vs. $\varrho$,
$\sigma$ vs. $\varsigma$.

\subsection{Uppercase Greek Letters}

\[
\Gamma, \Delta, \Theta, \Lambda, \Xi, \Pi, \Sigma, \Upsilon, \Phi, \Psi, \Omega
\]

Note: Some Greek letters like $A, B, E, Z, H, I, K, M, N, O, P, T, X$ are
identical to Latin letters and don't have special commands.

% ============================================================================
\section{Roots and Radicals}
% ============================================================================

Square roots use \verb|\sqrt|, and nth roots use \verb|\sqrt[n]|:

\[
\sqrt{2}, \quad \sqrt{x^2 + y^2}, \quad \sqrt[3]{27} = 3, \quad
\sqrt[n]{x^n} = x, \quad \sqrt{1 + \sqrt{1 + \sqrt{1 + \cdots}}}
\]

% ============================================================================
\section{Sums, Products, and Limits}
% ============================================================================

\subsection{Summation}

The \verb|\sum| command creates summation notation:

\[
\sum_{i=1}^n i = \frac{n(n+1)}{2}, \quad
\sum_{i=0}^\infty \frac{1}{2^i} = 2, \quad
\sum_{k=0}^n \binom{n}{k} x^k y^{n-k} = (x+y)^n
\]

In inline mode, limits appear beside the symbol: $\sum_{i=1}^n x_i$. To force
display-style limits in inline mode, use \verb|\limits|: $\sum\limits_{i=1}^n x_i$.

\subsection{Products}

The \verb|\prod| command creates product notation:

\[
\prod_{i=1}^n i = n!, \quad
\prod_{i=1}^n x_i, \quad
\prod_{p \text{ prime}} \left(1 - \frac{1}{p^2}\right) = \frac{6}{\pi^2}
\]

\subsection{Limits}

The \verb|\lim| command creates limit notation:

\[
\lim_{x \to 0} \frac{\sin x}{x} = 1, \quad
\lim_{n \to \infty} \left(1 + \frac{1}{n}\right)^n = e, \quad
\lim_{x \to a^+} f(x), \quad \lim_{x \to a^-} f(x)
\]

Other limit-like operators: $\limsup$, $\liminf$, $\sup$, $\inf$, $\max$, $\min$.

% ============================================================================
\section{Integrals}
% ============================================================================

\subsection{Single Integrals}

The \verb|\int| command creates integral signs:

\[
\int f(x) \, dx, \quad
\int_0^1 x^2 \, dx = \frac{1}{3}, \quad
\int_{-\infty}^{\infty} e^{-x^2} \, dx = \sqrt{\pi}
\]

Note the use of \verb|\,| for a thin space before $dx$.

\subsection{Multiple Integrals}

Multiple integrals use \verb|\iint|, \verb|\iiint|, etc.:

\[
\iint_D f(x,y) \, dA, \quad
\iiint_V f(x,y,z) \, dV, \quad
\oint_C \mathbf{F} \cdot d\mathbf{r}
\]

The \verb|\oint| command creates a closed path integral symbol.

\subsection{Definite Integrals with Evaluation Brackets}

\[
\left. \frac{x^3}{3} \right|_0^1 = \frac{1}{3} - 0 = \frac{1}{3}
\]

% ============================================================================
\section{Common Mathematical Functions}
% ============================================================================

Many common functions have predefined commands that format them correctly
(upright, with proper spacing):

\[
\sin x, \cos x, \tan x, \sec x, \csc x, \cot x
\]

\[
\arcsin x, \arccos x, \arctan x, \sinh x, \cosh x, \tanh x
\]

\[
\log x, \ln x, \lg x, \exp x
\]

\[
\det A, \dim V, \ker f, \deg p, \gcd(a,b), \lcm(a,b)
\]

Compare $\sin x$ (correct, upright) with $sin x$ (incorrect, italicized as
separate variables).

% ============================================================================
\section{Brackets and Delimiters}
% ============================================================================

\subsection{Manual Sizing}

Parentheses, brackets, and braces can be sized manually:

\[
( ), \quad [ ], \quad \{ \}, \quad | |, \quad \| \|, \quad \langle \rangle
\]

Manual sizing with \verb|\big|, \verb|\Big|, \verb|\bigg|, \verb|\Bigg|:

\[
( \big( \Big( \bigg( \Bigg(
\]

\subsection{Automatic Sizing}

Use \verb|\left| and \verb|\right| for automatic delimiter sizing:

\[
\left( \frac{x^2}{y^3} \right), \quad
\left[ \sum_{i=1}^n x_i \right]^2, \quad
\left\{ x \in \mathbb{R} : x^2 < 1 \right\}
\]

\[
\left| \frac{a}{b} \right|, \quad
\left\| \mathbf{v} \right\|, \quad
\left\langle u, v \right\rangle
\]

Every \verb|\left| must have a matching \verb|\right|. Use \verb|\right.| for
an invisible delimiter:

\[
\left. \frac{df}{dx} \right|_{x=0}
\]

% ============================================================================
\section{Matrices and Vectors}
% ============================================================================

Matrices use the \texttt{matrix} environment (or variants with delimiters):

\[
\begin{matrix}
a & b \\
c & d
\end{matrix}, \quad
\begin{pmatrix}
a & b \\
c & d
\end{pmatrix}, \quad
\begin{bmatrix}
a & b \\
c & d
\end{bmatrix}, \quad
\begin{vmatrix}
a & b \\
c & d
\end{vmatrix}
\]

Larger example:

\[
A = \begin{bmatrix}
1 & 2 & 3 \\
4 & 5 & 6 \\
7 & 8 & 9
\end{bmatrix}, \quad
\mathbf{v} = \begin{pmatrix}
x \\ y \\ z
\end{pmatrix}
\]

% ============================================================================
\section{Accents and Decorations}
% ============================================================================

Various accents and decorations are available:

\[
\hat{a}, \quad \bar{x}, \quad \tilde{n}, \quad \vec{v}, \quad \dot{x}, \quad
\ddot{y}, \quad \acute{a}, \quad \grave{a}, \quad \check{a}, \quad \breve{a}
\]

Wide accents for longer expressions:

\[
\widehat{xyz}, \quad \widetilde{abc}, \quad \overline{a+b+c}, \quad
\underline{x+y}
\]

Arrows above and below:

\[
\overrightarrow{AB}, \quad \overleftarrow{BA}, \quad
\overbrace{a + b + c}^{\text{sum}}, \quad
\underbrace{x + y + z}_{\text{variables}}
\]

% ============================================================================
\section{Special Symbols and Relations}
% ============================================================================

\subsection{Binary Relations}

\[
<, \leq, \ll, \quad >, \geq, \gg, \quad =, \neq, \equiv, \approx, \sim,
\simeq, \cong
\]

\[
\in, \notin, \ni, \quad \subset, \subseteq, \supset, \supseteq, \quad
\cup, \cap, \setminus
\]

\[
\parallel, \perp, \quad \vdash, \models, \quad \propto, \quad \asymp
\]

\subsection{Binary Operators}

\[
+, -, \times, \div, \pm, \mp, \cdot, *, \circ, \bullet, \oplus, \ominus,
\otimes, \odot
\]

\[
\wedge, \vee, \quad \cap, \cup, \quad \sqcap, \sqcup
\]

\subsection{Arrows}

\[
\leftarrow, \rightarrow, \leftrightarrow, \quad
\Leftarrow, \Rightarrow, \Leftrightarrow
\]

\[
\uparrow, \downarrow, \updownarrow, \quad
\nearrow, \searrow, \swarrow, \nwarrow
\]

\[
\longleftarrow, \longrightarrow, \implies, \iff, \mapsto
\]

\subsection{Miscellaneous Symbols}

\[
\infty, \quad \nabla, \quad \partial, \quad \forall, \quad \exists, \quad
\nexists, \quad \emptyset, \quad \varnothing, \quad \prime, \quad \hbar
\]

% ============================================================================
\section{Number Sets and Special Fonts}
% ============================================================================

Standard number sets use blackboard bold (\verb|\mathbb|):

\[
\mathbb{N}, \quad \mathbb{Z}, \quad \mathbb{Q}, \quad \mathbb{R}, \quad
\mathbb{C}, \quad \mathbb{H}
\]

Calligraphic letters (\verb|\mathcal|):

\[
\mathcal{A}, \mathcal{B}, \mathcal{C}, \mathcal{L}, \mathcal{O}, \mathcal{P}
\]

Script letters (\verb|\mathscr|, requires \texttt{mathrsfs} package):
% \[
% \mathscr{A}, \mathscr{B}, \mathscr{C}, \mathscr{L}
% \]

Fraktur letters (\verb|\mathfrak|):

\[
\mathfrak{A}, \mathfrak{B}, \mathfrak{C}, \mathfrak{g}, \mathfrak{h}
\]

% ============================================================================
\section{Spacing in Math Mode}
% ============================================================================

\LaTeX{} handles most spacing automatically, but manual spacing is sometimes
needed:

\begin{itemize}
    \item \verb|\,| -- thin space: $a\,b$
    \item \verb|\:| -- medium space: $a\:b$
    \item \verb|\;| -- thick space: $a\;b$
    \item \verb|\quad| -- 1em space: $a\quad b$
    \item \verb|\qquad| -- 2em space: $a\qquad b$
    \item \verb|\!| -- negative thin space: $\int\!\!\!\int$ vs $\int\int$
\end{itemize}

Example uses:
\[
f(x) \, dx \quad \text{(thin space before differential)}
\]
\[
\text{if } x > 0 \text{ then } y = x^2 \quad \text{(text mode for words)}
\]

% ============================================================================
\section{Text in Math Mode}
\label{sec:numbering}
% ============================================================================

To include normal text within math mode, use \verb|\text{...}|:

\[
f(x) = \begin{cases}
x^2 & \text{if } x \geq 0 \\
-x^2 & \text{if } x < 0
\end{cases}
\]

Compare: $f(x) = x^2 if x > 0$ (wrong) vs. $f(x) = x^2 \text{ if } x > 0$ (correct).

\end{document}

% ============================================================================
% COMPILATION TIP
% ============================================================================
% Compile with: pdflatex math_examples.tex
% For best results, compile twice to resolve cross-references.
% ============================================================================
