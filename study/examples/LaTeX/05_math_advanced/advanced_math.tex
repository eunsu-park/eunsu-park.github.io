% ============================================================================
% advanced_math.tex - Advanced Mathematical Typesetting
% ============================================================================
% This document demonstrates advanced mathematical features including matrices,
% cases, theorem environments, custom operators, and complex multi-line
% equation alignments.
%
% Compilation:
%   pdflatex advanced_math.tex (run twice for cross-references)
% ============================================================================

\documentclass[11pt]{article}

\usepackage{amsmath}
\usepackage{amssymb}
\usepackage{amsthm}
\usepackage{mathtools}   % Extension to amsmath

\usepackage[margin=1in]{geometry}
\usepackage[T1]{fontenc}
\usepackage{lmodern}

% For colored boxes and highlighting
\usepackage{xcolor}
\usepackage{tcolorbox}

% Graphics for diagrams
\usepackage{tikz}
\usetikzlibrary{matrix, arrows}

\usepackage{hyperref}
\hypersetup{colorlinks=true, linkcolor=blue}

% ============================================================================
% THEOREM ENVIRONMENTS
% ============================================================================

\theoremstyle{plain}
\newtheorem{theorem}{Theorem}[section]
\newtheorem{lemma}[theorem]{Lemma}
\newtheorem{proposition}[theorem]{Proposition}
\newtheorem{corollary}[theorem]{Corollary}

\theoremstyle{definition}
\newtheorem{definition}[theorem]{Definition}
\newtheorem{example}[theorem]{Example}

\theoremstyle{remark}
\newtheorem{remark}[theorem]{Remark}
\newtheorem{note}[theorem]{Note}

% Proof environment is already defined by amsthm

% ============================================================================
% CUSTOM OPERATORS AND COMMANDS
% ============================================================================

% Custom operators (formatted correctly with spacing)
\DeclareMathOperator{\Tr}{Tr}           % Trace
\DeclareMathOperator{\rank}{rank}       % Matrix rank
\DeclareMathOperator{\diag}{diag}       % Diagonal matrix
\DeclareMathOperator{\sgn}{sgn}         % Sign function
\DeclareMathOperator{\Span}{span}       % Linear span
\DeclareMathOperator{\im}{im}           % Image
\DeclareMathOperator*{\argmax}{arg\,max} % Argmax (with limits)
\DeclareMathOperator*{\argmin}{arg\,min} % Argmin (with limits)

% Custom commands
\newcommand{\R}{\mathbb{R}}
\newcommand{\C}{\mathbb{C}}
\newcommand{\N}{\mathbb{N}}
\newcommand{\Z}{\mathbb{Z}}
\newcommand{\Q}{\mathbb{Q}}

\newcommand{\norm}[1]{\left\| #1 \right\|}
\newcommand{\abs}[1]{\left| #1 \right|}
\newcommand{\ip}[2]{\left\langle #1, #2 \right\rangle}

\title{Advanced Mathematical Typesetting in \LaTeX}
\author{Mathematics Department}
\date{\today}

\begin{document}

\maketitle

\tableofcontents

\section{Introduction}

This document demonstrates advanced mathematical typesetting techniques in
\LaTeX, building upon basic mathematics to show matrices, theorem environments,
custom operators, and sophisticated alignment strategies.

% ============================================================================
\section{Matrices and Determinants}
% ============================================================================

\subsection{Basic Matrix Environments}

\LaTeX{} provides several matrix environments with different delimiters:

\begin{align*}
\begin{matrix}
a & b \\
c & d
\end{matrix} \quad
\begin{pmatrix}
a & b \\
c & d
\end{pmatrix} \quad
\begin{bmatrix}
a & b \\
c & d
\end{bmatrix} \quad
\begin{vmatrix}
a & b \\
c & d
\end{vmatrix} \quad
\begin{Vmatrix}
a & b \\
c & d
\end{Vmatrix}
\end{align*}

From left to right: \texttt{matrix}, \texttt{pmatrix}, \texttt{bmatrix},
\texttt{vmatrix} (determinant), \texttt{Vmatrix} (norm).

\subsection{Larger Matrices}

\begin{equation}
A = \begin{bmatrix}
a_{11} & a_{12} & a_{13} & \cdots & a_{1n} \\
a_{21} & a_{22} & a_{23} & \cdots & a_{2n} \\
\vdots & \vdots & \vdots & \ddots & \vdots \\
a_{m1} & a_{m2} & a_{m3} & \cdots & a_{mn}
\end{bmatrix}
\end{equation}

Identity matrix:
\begin{equation}
I_3 = \begin{pmatrix}
1 & 0 & 0 \\
0 & 1 & 0 \\
0 & 0 & 1
\end{pmatrix}
\end{equation}

\subsection{Special Matrix Types}

Diagonal matrix using custom operator:
\begin{equation}
D = \diag(d_1, d_2, \ldots, d_n) = \begin{pmatrix}
d_1 & 0 & \cdots & 0 \\
0 & d_2 & \cdots & 0 \\
\vdots & \vdots & \ddots & \vdots \\
0 & 0 & \cdots & d_n
\end{pmatrix}
\end{equation}

Block matrix:
\begin{equation}
M = \left[\begin{array}{c|c}
A & B \\
\hline
C & D
\end{array}\right]
\end{equation}

Augmented matrix for linear systems:
\begin{equation}
\left[\begin{array}{ccc|c}
1 & 2 & 3 & 4 \\
0 & 1 & 5 & 6 \\
0 & 0 & 1 & 7
\end{array}\right]
\end{equation}

\subsection{Matrix Operations}

\begin{theorem}[Matrix Multiplication]
\label{thm:matmul}
For matrices $A \in \R^{m \times n}$ and $B \in \R^{n \times p}$, the
product $C = AB \in \R^{m \times p}$ has entries:
\begin{equation}
c_{ij} = \sum_{k=1}^n a_{ik} b_{kj}
\end{equation}
\end{theorem}

\begin{example}
\begin{equation}
\begin{pmatrix}
1 & 2 \\
3 & 4
\end{pmatrix}
\begin{pmatrix}
5 & 6 \\
7 & 8
\end{pmatrix}
=
\begin{pmatrix}
19 & 22 \\
43 & 50
\end{pmatrix}
\end{equation}
\end{example}

% ============================================================================
\section{The \texttt{cases} Environment}
% ============================================================================

\subsection{Piecewise Functions}

The \texttt{cases} environment is ideal for piecewise definitions:

\begin{equation}
f(x) = \begin{cases}
x^2 & \text{if } x \geq 0 \\
-x^2 & \text{if } x < 0
\end{cases}
\end{equation}

\begin{example}[Heaviside Step Function]
\begin{equation}
H(x) = \begin{cases}
0 & \text{if } x < 0 \\
\frac{1}{2} & \text{if } x = 0 \\
1 & \text{if } x > 0
\end{cases}
\end{equation}
\end{example}

\subsection{Recursive Definitions}

The Fibonacci sequence:

\begin{equation}
F(n) = \begin{cases}
0 & \text{if } n = 0 \\
1 & \text{if } n = 1 \\
F(n-1) + F(n-2) & \text{if } n \geq 2
\end{cases}
\end{equation}

\subsection{Multiple Cases}

\begin{equation}
\text{grade}(x) = \begin{cases}
\text{A} & \text{if } 90 \leq x \leq 100 \\
\text{B} & \text{if } 80 \leq x < 90 \\
\text{C} & \text{if } 70 \leq x < 80 \\
\text{D} & \text{if } 60 \leq x < 70 \\
\text{F} & \text{if } 0 \leq x < 60
\end{cases}
\end{equation}

% ============================================================================
\section{Theorem Environments}
% ============================================================================

\subsection{Theorems, Lemmas, and Propositions}

\begin{definition}[Inner Product Space]
\label{def:inner_product}
An inner product space is a vector space $V$ over $\R$ or $\C$ equipped with
an inner product $\ip{\cdot}{\cdot}: V \times V \to \R$ (or $\C$) satisfying:
\begin{enumerate}
    \item Linearity: $\ip{ax + by}{z} = a\ip{x}{z} + b\ip{y}{z}$
    \item Symmetry: $\ip{x}{y} = \overline{\ip{y}{x}}$
    \item Positive definiteness: $\ip{x}{x} \geq 0$ with equality iff $x = 0$
\end{enumerate}
\end{definition}

\begin{lemma}[Cauchy-Schwarz Inequality]
\label{lem:cauchy_schwarz}
For any vectors $x, y$ in an inner product space:
\begin{equation}
\abs{\ip{x}{y}} \leq \norm{x} \norm{y}
\end{equation}
with equality if and only if $x$ and $y$ are linearly dependent.
\end{lemma}

\begin{theorem}[Spectral Theorem for Symmetric Matrices]
\label{thm:spectral}
Let $A \in \R^{n \times n}$ be a symmetric matrix. Then:
\begin{enumerate}
    \item All eigenvalues of $A$ are real
    \item Eigenvectors corresponding to distinct eigenvalues are orthogonal
    \item $A$ can be diagonalized as $A = Q\Lambda Q^T$ where $Q$ is orthogonal
    and $\Lambda$ is diagonal
\end{enumerate}
\end{theorem}

\begin{proof}
Let $\lambda$ be an eigenvalue with eigenvector $v$. Then $Av = \lambda v$.
Taking the conjugate transpose:
\begin{align}
v^* A^* &= \lambda^* v^* \\
v^* A &= \lambda^* v^* \quad \text{(since $A$ is symmetric)}
\end{align}
Multiplying the first equation on the left by $v^*$:
\begin{equation}
v^* A v = \lambda v^* v
\end{equation}
Multiplying the second on the right by $v$:
\begin{equation}
v^* A v = \lambda^* v^* v
\end{equation}
Therefore $\lambda = \lambda^*$, so $\lambda \in \R$.
\end{proof}

\begin{corollary}
\label{cor:eigenvalues_real}
Real symmetric matrices have real eigenvalues.
\end{corollary}

\begin{remark}
Theorem~\ref{thm:spectral} is fundamental in many applications including
principal component analysis (PCA) and quadratic form optimization.
\end{remark}

% ============================================================================
\section{Custom Operators}
% ============================================================================

\subsection{Declaring Math Operators}

Custom operators are declared in the preamble using
\verb|\DeclareMathOperator|:

\begin{verbatim}
\DeclareMathOperator{\Tr}{Tr}
\DeclareMathOperator{\rank}{rank}
\DeclareMathOperator*{\argmax}{arg\,max}
\end{verbatim}

The starred version \verb|\DeclareMathOperator*| allows limits above and below.

\subsection{Usage Examples}

Trace of a matrix:
\begin{equation}
\Tr(A) = \sum_{i=1}^n a_{ii}
\end{equation}

Matrix rank:
\begin{equation}
\rank(A) \leq \min(m, n) \quad \text{for } A \in \R^{m \times n}
\end{equation}

Argmax with limits:
\begin{equation}
\theta^* = \argmax_{\theta \in \Theta} \mathcal{L}(\theta; \mathcal{D})
\end{equation}

Sign function:
\begin{equation}
\sgn(x) = \begin{cases}
-1 & \text{if } x < 0 \\
0 & \text{if } x = 0 \\
1 & \text{if } x > 0
\end{cases}
\end{equation}

% ============================================================================
\section{Multi-Line Equations with Complex Alignment}
% ============================================================================

\subsection{The \texttt{align} Environment}

\begin{align}
(a + b)^3 &= (a + b)(a + b)^2 \\
          &= (a + b)(a^2 + 2ab + b^2) \\
          &= a^3 + 2a^2b + ab^2 + a^2b + 2ab^2 + b^3 \\
          &= a^3 + 3a^2b + 3ab^2 + b^3
\end{align}

\subsection{The \texttt{gather} Environment}

For equations that don't need alignment:

\begin{gather}
\nabla \cdot \mathbf{E} = \frac{\rho}{\epsilon_0} \\
\nabla \cdot \mathbf{B} = 0 \\
\nabla \times \mathbf{E} = -\frac{\partial \mathbf{B}}{\partial t} \\
\nabla \times \mathbf{B} = \mu_0\mathbf{J} +
\mu_0\epsilon_0\frac{\partial \mathbf{E}}{\partial t}
\end{gather}

\subsection{The \texttt{multline} Environment}

For a single long equation:

\begin{multline}
f(x_1, x_2, \ldots, x_n) = x_1^2 + x_2^2 + \cdots + x_n^2 \\
+ 2x_1x_2 + 2x_1x_3 + \cdots + 2x_1x_n \\
+ 2x_2x_3 + 2x_2x_4 + \cdots + 2x_2x_n \\
+ \cdots + 2x_{n-1}x_n
\end{multline}

\subsection{Aligned Equations with Conditions}

\begin{equation}
\begin{aligned}
\text{minimize} \quad & \frac{1}{2}\norm{x}^2 \\
\text{subject to} \quad & Ax = b \\
                        & x \geq 0
\end{aligned}
\end{equation}

Or using \texttt{array} for more control:

\begin{equation}
\begin{array}{ll}
\text{maximize} & c^T x \\
\text{subject to} & Ax \leq b \\
                  & x \geq 0
\end{array}
\end{equation}

% ============================================================================
\section{Advanced Equation Formatting}
% ============================================================================

\subsection{Underbrace and Overbrace}

\begin{equation}
\overbrace{a + b + c}^{\text{sum}} = \underbrace{d + e + f}_{\text{also sum}}
\end{equation}

\begin{equation}
z = \overbrace{
    \underbrace{x}_{\text{real}} + i\underbrace{y}_{\text{imaginary}}
}^{\text{complex number}}
\end{equation}

\subsection{Stacked Relations}

\begin{equation}
A \xleftarrow[\text{row operations}]{\text{elementary}} B
\xrightarrow[\text{column operations}]{\text{elementary}} C
\end{equation}

Using \texttt{mathtools} package:

\begin{equation}
f(n) \xlongequal{\text{def}} \sum_{i=0}^n i^2
\end{equation}

\subsection{Text in Math Mode}

\begin{equation}
P(A \mid B) = \frac{P(B \mid A) \cdot P(A)}{P(B)}
\quad \text{where } P(B) \neq 0
\end{equation}

\subsection{Multi-Line Subscripts and Superscripts}

\begin{equation}
\sum_{\substack{0 \leq i \leq n \\ 0 \leq j \leq m}} a_{ij}
\end{equation}

\begin{equation}
\prod_{\substack{p \text{ prime} \\ p < 100}} p
\end{equation}

% ============================================================================
\section{Commutative Diagrams (Optional)}
% ============================================================================

Using TikZ for commutative diagrams:

\begin{equation}
\begin{tikzcd}
A \arrow[r, "f"] \arrow[d, "g"'] & B \arrow[d, "h"] \\
C \arrow[r, "k"'] & D
\end{tikzcd}
\end{equation}

% ============================================================================
\section{Complex Example: Lagrange Multipliers}
% ============================================================================

\begin{theorem}[Method of Lagrange Multipliers]
\label{thm:lagrange}
Let $f: \R^n \to \R$ and $g: \R^n \to \R$ be continuously differentiable.
To find extrema of $f$ subject to the constraint $g(x) = 0$, solve:
\begin{equation}
\nabla f(x^*) = \lambda \nabla g(x^*)
\end{equation}
where $\lambda$ is the Lagrange multiplier.
\end{theorem}

\begin{example}
Find the maximum and minimum of $f(x,y) = xy$ subject to $x^2 + y^2 = 1$.

\begin{align}
\text{Lagrangian: } \mathcal{L}(x, y, \lambda) &= xy - \lambda(x^2 + y^2 - 1)
\end{align}

Setting partial derivatives to zero:
\begin{align}
\frac{\partial \mathcal{L}}{\partial x} &= y - 2\lambda x = 0 \\
\frac{\partial \mathcal{L}}{\partial y} &= x - 2\lambda y = 0 \\
\frac{\partial \mathcal{L}}{\partial \lambda} &= -(x^2 + y^2 - 1) = 0
\end{align}

From the first two equations: $y = 2\lambda x$ and $x = 2\lambda y$, so:
\begin{equation}
x = 2\lambda(2\lambda x) = 4\lambda^2 x
\end{equation}

If $x \neq 0$: $1 = 4\lambda^2$, so $\lambda = \pm \frac{1}{2}$.

Critical points occur at $\left(\pm\frac{1}{\sqrt{2}}, \pm\frac{1}{\sqrt{2}}\right)$
with values:
\begin{equation}
f\left(\frac{1}{\sqrt{2}}, \frac{1}{\sqrt{2}}\right) = \frac{1}{2} \quad
\text{(maximum)}
\end{equation}
\begin{equation}
f\left(\frac{1}{\sqrt{2}}, -\frac{1}{\sqrt{2}}\right) = -\frac{1}{2} \quad
\text{(minimum)}
\end{equation}
\end{example}

% ============================================================================
\section{Conclusion}
% ============================================================================

This document has demonstrated advanced mathematical typesetting including:

\begin{itemize}
    \item Matrix environments with various delimiters
    \item Piecewise functions using \texttt{cases}
    \item Theorem-like environments (theorem, lemma, proof, etc.)
    \item Custom mathematical operators
    \item Multi-line equation alignment with \texttt{align}, \texttt{gather},
    \texttt{multline}
    \item Advanced formatting techniques (underbrace, overbrace, stacked
    relations)
\end{itemize}

With these tools, you can typeset virtually any mathematical content with
professional quality.

\end{document}

% ============================================================================
% PACKAGE REQUIREMENTS
% ============================================================================
% This document requires:
%   - amsmath, amssymb, amsthm (standard math packages)
%   - mathtools (extension to amsmath)
%   - tikz-cd (for commutative diagrams, optional)
%
% Compile with: pdflatex advanced_math.tex (twice for cross-references)
% ============================================================================
