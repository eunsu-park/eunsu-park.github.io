% ============================================================================
% figure_demo.tex - Figures, Graphics, and Image Inclusion
% ============================================================================
% This document demonstrates how to include figures, use subfigures, add
% captions and labels, and cross-reference figures in LaTeX.
%
% Compilation:
%   pdflatex figure_demo.tex (run twice for cross-references)
%
% Note: This example uses 'example-image' placeholder images that are
% provided by the mwe package. For your own documents, replace these
% with actual image files.
% ============================================================================

\documentclass[11pt]{article}

\usepackage[margin=1in]{geometry}
\usepackage[T1]{fontenc}
\usepackage{lmodern}

% Graphics packages
\usepackage{graphicx}      % For \includegraphics
\usepackage{float}         % For H placement specifier
\usepackage{caption}       % Enhanced caption formatting
\usepackage{subcaption}    % For subfigures

% Example images (part of mwe package, automatically available)
% For your documents, you'll use actual image files like:
% \includegraphics{path/to/your/image.png}

% Colors for boxes
\usepackage{xcolor}
\usepackage{tcolorbox}

% Hyperlinks
\usepackage{hyperref}
\hypersetup{colorlinks=true, linkcolor=blue}

% Set default graphics path (optional)
% \graphicspath{{./images/}{./figures/}}

\title{Figures and Graphics in \LaTeX}
\author{Graphics Department}
\date{\today}

\begin{document}

\maketitle

\tableofcontents

\listoffigures

\section{Introduction}

This document demonstrates how to include and manage figures in \LaTeX{}
documents. Figures are floating environments that can contain images, diagrams,
or other graphical content.

% ============================================================================
\section{Basic Figure Inclusion}
% ============================================================================

\subsection{The \texttt{figure} Environment}

The basic structure for including a figure is:

\begin{verbatim}
\begin{figure}[placement]
    \centering
    \includegraphics[options]{filename}
    \caption{Caption text}
    \label{fig:label}
\end{figure}
\end{verbatim}

\subsection{Simple Figure Example}

Here is a simple figure using a placeholder image:

\begin{figure}[h]
    \centering
    \includegraphics[width=0.5\textwidth]{example-image}
    \caption{A simple example figure with default placeholder image}
    \label{fig:simple}
\end{figure}

Figure~\ref{fig:simple} shows a basic example. The \verb|[h]| placement
specifier suggests placing the figure "here" if possible.

\subsection{Placement Specifiers}

LaTeX uses placement specifiers to control where figures appear:

\begin{table}[h]
\centering
\begin{tabular}{cl}
\hline
Specifier & Meaning \\
\hline
\texttt{h} & Here (approximately) \\
\texttt{t} & Top of page \\
\texttt{b} & Bottom of page \\
\texttt{p} & On a separate page of floats \\
\texttt{!} & Override LaTeX's placement rules \\
\texttt{H} & Exactly here (requires \texttt{float} package) \\
\hline
\end{tabular}
\end{table}

You can combine specifiers, e.g., \verb|[htbp]| tries here, then top, then
bottom, then a float page.

% ============================================================================
\section{Sizing and Scaling Images}
% ============================================================================

\subsection{Width-Based Sizing}

The most common approach is to specify width relative to text width:

\begin{figure}[h]
    \centering
    \includegraphics[width=0.3\textwidth]{example-image-a}
    \caption{Image scaled to 30\% of text width}
    \label{fig:width30}
\end{figure}

\begin{figure}[h]
    \centering
    \includegraphics[width=0.7\textwidth]{example-image-b}
    \caption{Image scaled to 70\% of text width}
    \label{fig:width70}
\end{figure}

\subsection{Height-Based Sizing}

You can also specify height:

\begin{figure}[h]
    \centering
    \includegraphics[height=3cm]{example-image-c}
    \caption{Image with fixed height of 3cm}
    \label{fig:height}
\end{figure}

\subsection{Scale Factor}

Or use a scale factor:

\begin{figure}[h]
    \centering
    \includegraphics[scale=0.5]{example-image}
    \caption{Image scaled to 50\% of original size}
    \label{fig:scale}
\end{figure}

\subsection{Maintaining Aspect Ratio}

By default, specifying only width or height maintains the aspect ratio.
To specify both (distorting the image):

\begin{verbatim}
\includegraphics[width=5cm, height=3cm]{image}
\end{verbatim}

To maintain aspect ratio while fitting in a box:

\begin{verbatim}
\includegraphics[width=5cm, height=3cm, keepaspectratio]{image}
\end{verbatim}

% ============================================================================
\section{Captions and Labels}
% ============================================================================

\subsection{Caption Placement}

Captions can appear above or below the figure:

\begin{figure}[h]
    \centering
    \caption{Caption above the image}
    \includegraphics[width=0.4\textwidth]{example-image}
    \label{fig:caption_top}
\end{figure}

\begin{figure}[h]
    \centering
    \includegraphics[width=0.4\textwidth]{example-image}
    \caption{Caption below the image (more common)}
    \label{fig:caption_bottom}
\end{figure}

Convention: captions typically appear below figures but above tables.

\subsection{Long Captions}

For long captions, use the optional short caption for the list of figures:

\begin{figure}[h]
    \centering
    \includegraphics[width=0.5\textwidth]{example-image}
    \caption[Short caption for LOF]{This is a very long caption that provides
    detailed information about the figure, including methodology, data sources,
    and interpretation. It may span multiple lines and include technical details
    that would clutter the List of Figures. The short caption in brackets
    appears in the LOF instead.}
    \label{fig:long_caption}
\end{figure}

\subsection{Cross-Referencing}

Reference figures using \verb|\ref| or \verb|\pageref|:

\begin{itemize}
    \item Figure~\ref{fig:simple} shows a simple example
    \item See Figure~\ref{fig:width70} on page~\pageref{fig:width70}
    \item As demonstrated in Figures~\ref{fig:caption_top} and~\ref{fig:caption_bottom}
\end{itemize}

% ============================================================================
\section{Subfigures}
% ============================================================================

\subsection{Basic Subfigures}

The \texttt{subcaption} package provides the \texttt{subfigure} environment:

\begin{figure}[h]
    \centering
    \begin{subfigure}{0.45\textwidth}
        \centering
        \includegraphics[width=\textwidth]{example-image-a}
        \caption{First subfigure}
        \label{fig:sub_a}
    \end{subfigure}
    \hfill
    \begin{subfigure}{0.45\textwidth}
        \centering
        \includegraphics[width=\textwidth]{example-image-b}
        \caption{Second subfigure}
        \label{fig:sub_b}
    \end{subfigure}
    \caption{Two subfigures side by side}
    \label{fig:two_subfigs}
\end{figure}

Reference individual subfigures: Figure~\ref{fig:sub_a} shows the first image,
while Figure~\ref{fig:sub_b} shows the second. Together, they form
Figure~\ref{fig:two_subfigs}.

\subsection{Multiple Rows of Subfigures}

\begin{figure}[h]
    \centering
    \begin{subfigure}{0.3\textwidth}
        \centering
        \includegraphics[width=\textwidth]{example-image-a}
        \caption{Subfigure A}
        \label{fig:multi_a}
    \end{subfigure}
    \hfill
    \begin{subfigure}{0.3\textwidth}
        \centering
        \includegraphics[width=\textwidth]{example-image-b}
        \caption{Subfigure B}
        \label{fig:multi_b}
    \end{subfigure}
    \hfill
    \begin{subfigure}{0.3\textwidth}
        \centering
        \includegraphics[width=\textwidth]{example-image-c}
        \caption{Subfigure C}
        \label{fig:multi_c}
    \end{subfigure}

    \vspace{0.5cm}

    \begin{subfigure}{0.45\textwidth}
        \centering
        \includegraphics[width=\textwidth]{example-image}
        \caption{Subfigure D}
        \label{fig:multi_d}
    \end{subfigure}
    \hfill
    \begin{subfigure}{0.45\textwidth}
        \centering
        \includegraphics[width=\textwidth]{example-image-golden}
        \caption{Subfigure E}
        \label{fig:multi_e}
    \end{subfigure}

    \caption{Multiple subfigures arranged in rows}
    \label{fig:multi_subfigs}
\end{figure}

Figure~\ref{fig:multi_subfigs} contains five subfigures arranged in two rows.

\subsection{Shared Subfigure Captions}

You can also use \texttt{subcaptionbox} for more compact code:

\begin{figure}[h]
    \centering
    \subcaptionbox{First image\label{fig:box_a}}
        {\includegraphics[width=0.3\textwidth]{example-image-a}}
    \hfill
    \subcaptionbox{Second image\label{fig:box_b}}
        {\includegraphics[width=0.3\textwidth]{example-image-b}}
    \hfill
    \subcaptionbox{Third image\label{fig:box_c}}
        {\includegraphics[width=0.3\textwidth]{example-image-c}}
    \caption{Three images using subcaptionbox}
    \label{fig:subcaptionbox}
\end{figure}

% ============================================================================
\section{Advanced Figure Formatting}
% ============================================================================

\subsection{Rotating Images}

Rotate images using the \texttt{angle} option:

\begin{figure}[h]
    \centering
    \includegraphics[width=0.3\textwidth, angle=45]{example-image}
    \caption{Image rotated 45 degrees}
    \label{fig:rotated}
\end{figure}

\subsection{Clipping and Trimming}

Crop images using \texttt{trim} and \texttt{clip}:

\begin{verbatim}
\includegraphics[trim={left bottom right top}, clip, width=5cm]{image}
\end{verbatim}

\subsection{Framed Figures}

Add a frame around figures:

\begin{figure}[h]
    \centering
    \fbox{\includegraphics[width=0.4\textwidth]{example-image}}
    \caption{Image with a frame}
    \label{fig:framed}
\end{figure}

\subsection{Custom Frame Thickness}

\begin{figure}[h]
    \centering
    \setlength{\fboxrule}{3pt}
    \fbox{\includegraphics[width=0.4\textwidth]{example-image}}
    \caption{Image with thick frame (3pt)}
    \label{fig:thick_frame}
\end{figure}

% ============================================================================
\section{Figure Positioning Tips}
% ============================================================================

\begin{tcolorbox}[colback=blue!5, colframe=blue!75!black, title=Best Practices]
\begin{enumerate}
    \item Use \verb|[htbp]| as default placement specifiers
    \item Prefer \verb|\textwidth| for width scaling (responsive to document layout)
    \item Always include \verb|\centering| before \verb|\includegraphics|
    \item Place \verb|\label| after \verb|\caption| for correct referencing
    \item Use vector formats (PDF, SVG) over raster (PNG, JPG) when possible
    \item Keep figure files organized in a separate directory
\end{enumerate}
\end{tcolorbox}

\subsection{Forcing Exact Placement}

To force a figure to appear exactly at the current position (not recommended
for most cases):

\begin{figure}[H]
    \centering
    \includegraphics[width=0.3\textwidth]{example-image}
    \caption{Figure with [H] placement (exactly here, not floating)}
    \label{fig:exact_here}
\end{figure}

The \verb|[H]| specifier requires the \texttt{float} package and prevents
the figure from floating. This can lead to large whitespace gaps.

\subsection{Preventing Figures from Moving Too Far}

Control float placement globally in the preamble:

\begin{verbatim}
\setcounter{topnumber}{2}         % Max floats at top
\setcounter{bottomnumber}{2}      % Max floats at bottom
\setcounter{totalnumber}{4}       % Max floats per page
\renewcommand{\topfraction}{0.7}  % Max fraction of page for top floats
\renewcommand{\bottomfraction}{0.3}
\renewcommand{\textfraction}{0.2} % Min fraction for text
\renewcommand{\floatpagefraction}{0.7} % Min for float-only page
\end{verbatim}

% ============================================================================
\section{Supported Image Formats}
% ============================================================================

\subsection{Format Recommendations}

\begin{table}[h]
\centering
\caption{Image format recommendations for LaTeX}
\begin{tabular}{llp{6cm}}
\hline
Format & Type & Best Use \\
\hline
PDF & Vector & Diagrams, plots, line art \\
EPS & Vector & Legacy format (still widely used) \\
PNG & Raster & Screenshots, photos with transparency \\
JPG & Raster & Photographs (no transparency) \\
SVG & Vector & Web graphics (convert to PDF first) \\
\hline
\end{tabular}
\end{table}

\subsection{Compilation Differences}

\begin{itemize}
    \item \textbf{pdflatex}: Supports PDF, PNG, JPG (not EPS directly)
    \item \textbf{latex + dvips + ps2pdf}: Supports EPS (not PDF directly)
    \item \textbf{lualatex/xelatex}: Supports PDF, PNG, JPG
\end{itemize}

For maximum compatibility, convert EPS to PDF:
\begin{verbatim}
epstopdf myimage.eps
\end{verbatim}

% ============================================================================
\section{Complete Example}
% ============================================================================

Here is a complete example combining multiple techniques:

\begin{figure}[htbp]
    \centering

    \begin{subfigure}{0.48\textwidth}
        \centering
        \includegraphics[width=\textwidth]{example-image-a}
        \caption{Original experimental data showing baseline measurements}
        \label{fig:complete_original}
    \end{subfigure}
    \hfill
    \begin{subfigure}{0.48\textwidth}
        \centering
        \includegraphics[width=\textwidth]{example-image-b}
        \caption{Processed results after applying the proposed algorithm}
        \label{fig:complete_processed}
    \end{subfigure}

    \vspace{0.5cm}

    \begin{subfigure}{0.6\textwidth}
        \centering
        \includegraphics[width=\textwidth]{example-image-c}
        \caption{Comparative analysis demonstrating 25\% improvement over
        baseline}
        \label{fig:complete_analysis}
    \end{subfigure}

    \caption[Complete experimental results]{Complete experimental results.
    \textbf{(a)} Shows the original data collected from the experiment.
    \textbf{(b)} Displays the processed output. \textbf{(c)} Presents the
    comparative analysis with statistical significance ($p < 0.05$).}
    \label{fig:complete}
\end{figure}

As shown in Figure~\ref{fig:complete}, the experimental procedure yielded
significant results. The original data (Figure~\ref{fig:complete_original})
was processed to obtain Figure~\ref{fig:complete_processed}, and the
comparative analysis in Figure~\ref{fig:complete_analysis} confirms our
hypothesis.

% ============================================================================
\section{Conclusion}
% ============================================================================

This document has demonstrated:

\begin{itemize}
    \item Basic figure inclusion with \verb|\includegraphics|
    \item Sizing and scaling options
    \item Captions and cross-referencing
    \item Creating subfigures with \texttt{subcaption}
    \item Advanced formatting (rotation, frames, clipping)
    \item Placement control and best practices
    \item Image format recommendations
\end{itemize}

For your own documents, remember to:
\begin{enumerate}
    \item Replace \texttt{example-image} with your actual image files
    \item Set \verb|\graphicspath| to your image directory
    \item Use vector formats (PDF) for diagrams and plots
    \item Always compile twice to resolve cross-references
\end{enumerate}

\end{document}

% ============================================================================
% COMPILATION NOTES
% ============================================================================
% 1. Compile with: pdflatex figure_demo.tex (twice for cross-references)
%
% 2. For your own images, create a structure like:
%    project/
%    ├── document.tex
%    └── images/
%        ├── figure1.pdf
%        ├── figure2.png
%        └── photo.jpg
%
% 3. Then add to preamble:
%    \graphicspath{{./images/}}
%
% 4. And use:
%    \includegraphics{figure1}  (no extension needed)
% ============================================================================
