% ============================================================================
% tables_demo.tex - Tables in LaTeX
% ============================================================================
% This document demonstrates comprehensive table creation in LaTeX, including
% basic tabular environments, professional formatting with booktabs, multi-row
% and multi-column cells, colored rows, and numerical data with siunitx.
%
% Compilation:
%   pdflatex tables_demo.tex (run twice for cross-references)
% ============================================================================

\documentclass[11pt]{article}

\usepackage[margin=1in]{geometry}
\usepackage[T1]{fontenc}
\usepackage{lmodem}

% Table packages
\usepackage{array}         % Extended column specifications
\usepackage{tabularx}      % Auto-width columns
\usepackage{booktabs}      % Professional table formatting
\usepackage{multirow}      % Multi-row cells
\usepackage{makecell}      % Line breaks in cells
\usepackage{longtable}     % Tables spanning multiple pages

% Colors for table rows
\usepackage[table]{xcolor}

% Number formatting
\usepackage{siunitx}       % Align numbers on decimal point

% For better captions
\usepackage{caption}

% Hyperlinks
\usepackage{hyperref}
\hypersetup{colorlinks=true, linkcolor=blue}

\title{Tables in \LaTeX}
\author{Typesetting Department}
\date{\today}

\begin{document}

\maketitle

\tableofcontents

\listoftables

\section{Introduction}

Tables are essential for presenting structured data. This document demonstrates
various techniques for creating professional-looking tables in \LaTeX, from
basic \texttt{tabular} environments to advanced formatting.

% ============================================================================
\section{Basic Tables}
% ============================================================================

\subsection{The \texttt{tabular} Environment}

The basic table structure uses the \texttt{tabular} environment:

\begin{verbatim}
\begin{tabular}{column specifications}
    cell 1 & cell 2 & cell 3 \\
    cell 4 & cell 5 & cell 6 \\
\end{tabular}
\end{verbatim}

\subsection{Simple Table Example}

\begin{table}[h]
\centering
\caption{A basic table}
\label{tab:basic}
\begin{tabular}{lll}
\hline
Name & Age & City \\
\hline
Alice & 25 & New York \\
Bob & 30 & London \\
Charlie & 35 & Tokyo \\
\hline
\end{tabular}
\end{table}

Table~\ref{tab:basic} shows a simple three-column table with horizontal lines.

\subsection{Column Specifications}

\begin{table}[h]
\centering
\caption{Column specification examples}
\label{tab:columns}
\begin{tabular}{|l|c|r|}
\hline
Left & Center & Right \\
\hline
Text & Text & Text \\
Aligned & Aligned & Aligned \\
Left & Center & Right \\
\hline
\end{tabular}
\end{table}

Common column types:
\begin{itemize}
    \item \texttt{l} -- left-aligned
    \item \texttt{c} -- centered
    \item \texttt{r} -- right-aligned
    \item \texttt{p\{width\}} -- paragraph column with specified width
    \item \texttt{|} -- vertical line
\end{itemize}

% ============================================================================
\section{Professional Tables with \texttt{booktabs}}
% ============================================================================

\subsection{Why Use \texttt{booktabs}?}

The \texttt{booktabs} package provides professional-looking horizontal rules
without vertical lines, which is considered better typographic practice.

\begin{table}[h]
\centering
\caption{Professional table with booktabs}
\label{tab:booktabs}
\begin{tabular}{lcc}
\toprule
Algorithm & Accuracy (\%) & Time (ms) \\
\midrule
Method A & 85.3 & 120 \\
Method B & 92.1 & 95 \\
Method C & 88.7 & 110 \\
\bottomrule
\end{tabular}
\end{table}

Key \texttt{booktabs} commands:
\begin{itemize}
    \item \verb|\toprule| -- top rule
    \item \verb|\midrule| -- middle rule (separates header from data)
    \item \verb|\bottomrule| -- bottom rule
    \item \verb|\cmidrule{i-j}| -- partial rule from column i to j
\end{itemize}

\subsection{Table with Partial Rules}

\begin{table}[h]
\centering
\caption{Using partial rules with cmidrule}
\label{tab:cmidrule}
\begin{tabular}{lrrrr}
\toprule
& \multicolumn{2}{c}{Method 1} & \multicolumn{2}{c}{Method 2} \\
\cmidrule(lr){2-3} \cmidrule(lr){4-5}
Dataset & Acc & Time & Acc & Time \\
\midrule
Set A & 85.3 & 120 & 87.2 & 115 \\
Set B & 90.1 & 95 & 91.5 & 90 \\
Set C & 88.7 & 110 & 89.3 & 105 \\
\bottomrule
\end{tabular}
\end{table}

% ============================================================================
\section{Multi-row and Multi-column Cells}
% ============================================================================

\subsection{Multi-column Cells}

Use \verb|\multicolumn{n}{alignment}{content}| to span multiple columns:

\begin{table}[h]
\centering
\caption{Table with multicolumn cells}
\label{tab:multicolumn}
\begin{tabular}{lccc}
\toprule
& \multicolumn{3}{c}{Measurements} \\
\cmidrule{2-4}
Sample & Trial 1 & Trial 2 & Trial 3 \\
\midrule
A & 12.5 & 12.7 & 12.3 \\
B & 15.2 & 15.4 & 15.1 \\
C & 18.9 & 18.7 & 19.0 \\
\midrule
\multicolumn{4}{c}{Average: 15.4} \\
\bottomrule
\end{tabular}
\end{table}

\subsection{Multi-row Cells}

Use \verb|\multirow{n}{width}{content}| to span multiple rows:

\begin{table}[h]
\centering
\caption{Table with multirow cells}
\label{tab:multirow}
\begin{tabular}{llrr}
\toprule
Category & Subcategory & Value 1 & Value 2 \\
\midrule
\multirow{3}{*}{Group A} & Item 1 & 10 & 20 \\
                         & Item 2 & 15 & 25 \\
                         & Item 3 & 12 & 22 \\
\midrule
\multirow{2}{*}{Group B} & Item 4 & 18 & 28 \\
                         & Item 5 & 20 & 30 \\
\bottomrule
\end{tabular}
\end{table}

\subsection{Combining Multi-row and Multi-column}

\begin{table}[h]
\centering
\caption{Complex table with multirow and multicolumn}
\label{tab:complex}
\begin{tabular}{llcccc}
\toprule
& & \multicolumn{2}{c}{Condition 1} & \multicolumn{2}{c}{Condition 2} \\
\cmidrule(lr){3-4} \cmidrule(lr){5-6}
Type & Subtype & Mean & SD & Mean & SD \\
\midrule
\multirow{3}{*}{Type A} & Var 1 & 85.3 & 2.1 & 87.2 & 1.8 \\
                        & Var 2 & 90.1 & 1.5 & 91.5 & 1.3 \\
                        & Var 3 & 88.7 & 1.9 & 89.3 & 1.7 \\
\midrule
\multirow{2}{*}{Type B} & Var 1 & 92.5 & 1.2 & 94.1 & 1.0 \\
                        & Var 2 & 95.3 & 0.9 & 96.2 & 0.8 \\
\bottomrule
\end{tabular}
\end{table}

% ============================================================================
\section{Colored Tables}
% ============================================================================

\subsection{Row Colors}

The \texttt{xcolor} package with \texttt{[table]} option enables row coloring:

\begin{table}[h]
\centering
\caption{Table with alternating row colors}
\label{tab:rowcolors}
\rowcolors{2}{gray!10}{white}
\begin{tabular}{lrrr}
\toprule
Product & Q1 Sales & Q2 Sales & Q3 Sales \\
\midrule
Product A & 1250 & 1380 & 1420 \\
Product B & 980 & 1050 & 1120 \\
Product C & 1560 & 1620 & 1590 \\
Product D & 2100 & 2250 & 2340 \\
Product E & 1840 & 1920 & 1980 \\
\bottomrule
\end{tabular}
\end{table}

\subsection{Column and Cell Colors}

\begin{table}[h]
\centering
\caption{Table with colored columns and cells}
\label{tab:cellcolors}
\begin{tabular}{l>{\columncolor{blue!10}}c>{\columncolor{green!10}}c}
\toprule
Item & Score 1 & Score 2 \\
\midrule
A & 85 & 90 \\
B & 78 & \cellcolor{red!20}65 \\
C & 92 & 88 \\
D & \cellcolor{yellow!30}95 & 93 \\
\bottomrule
\end{tabular}
\end{table}

% ============================================================================
\section{Numerical Data with \texttt{siunitx}}
% ============================================================================

\subsection{Decimal Alignment}

The \texttt{siunitx} package provides the \texttt{S} column type for aligning
numbers on the decimal point:

\begin{table}[h]
\centering
\caption{Numbers aligned with siunitx}
\label{tab:siunitx}
\begin{tabular}{lSSS}
\toprule
{Measurement} & {Value 1} & {Value 2} & {Value 3} \\
\midrule
Alpha & 12.345 & 1.2 & 0.00123 \\
Beta & 1234.5 & 123.45 & 12.345 \\
Gamma & 0.123 & 1234567.89 & 9.87654 \\
Delta & 999.999 & 1.00001 & 100.0 \\
\bottomrule
\end{tabular}
\end{table}

Note: Column headers in \texttt{S} columns must be enclosed in braces.

\subsection{Scientific Notation}

\begin{table}[h]
\centering
\caption{Scientific notation with siunitx}
\label{tab:scientific}
\begin{tabular}{lS[scientific-notation=true]}
\toprule
{Constant} & {Value} \\
\midrule
Speed of light & 2.998e8 \\
Planck constant & 6.626e-34 \\
Avogadro number & 6.022e23 \\
Electron mass & 9.109e-31 \\
\bottomrule
\end{tabular}
\end{table}

\subsection{Number Formatting Options}

\begin{table}[h]
\centering
\caption{Custom number formatting with siunitx}
\label{tab:formatting}
\begin{tabular}{
    l
    S[table-format=2.3]
    S[table-format=4.0, group-separator={,}]
    S[table-format=1.2e2]
}
\toprule
{Item} & {Precision} & {Large Numbers} & {Scientific} \\
\midrule
A & 12.345 & 12345 & 1.23e5 \\
B & 9.876 & 987654 & 9.88e5 \\
C & 45.123 & 4567 & 4.57e3 \\
\bottomrule
\end{tabular}
\end{table}

% ============================================================================
\section{Wide Tables}
% ============================================================================

\subsection{Using \texttt{tabularx}}

For tables that should fill the text width:

\begin{table}[h]
\centering
\caption{Table with auto-width columns using tabularx}
\label{tab:tabularx}
\begin{tabularx}{\textwidth}{lXX}
\toprule
Category & Description & Example \\
\midrule
Type A & This is a longer description that will automatically wrap to fill
the available space in the column. The X column type distributes space evenly.
& Example content that also wraps automatically. \\
Type B & Another description with significant content that demonstrates the
automatic width adjustment feature of tabularx. & More example content here. \\
Type C & Short description. & Short example. \\
\bottomrule
\end{tabularx}
\end{table}

\subsection{Landscape Tables}

For very wide tables, use the \texttt{rotating} package (not shown here, but
code example):

\begin{verbatim}
\usepackage{rotating}

\begin{sidewaystable}
\begin{tabular}{...}
...
\end{tabular}
\end{sidewaystable}
\end{verbatim}

% ============================================================================
\section{Long Tables Spanning Multiple Pages}
% ============================================================================

The \texttt{longtable} package allows tables to break across pages:

\begin{longtable}{llrr}
\caption{Long table example spanning multiple pages} \label{tab:long} \\
\toprule
ID & Name & Value 1 & Value 2 \\
\midrule
\endfirsthead

\multicolumn{4}{c}{{\tablename\ \thetable{} -- continued from previous page}} \\
\toprule
ID & Name & Value 1 & Value 2 \\
\midrule
\endhead

\midrule
\multicolumn{4}{r}{{Continued on next page}} \\
\endfoot

\bottomrule
\endlastfoot

% Table content
001 & Sample A & 12.5 & 15.3 \\
002 & Sample B & 14.2 & 16.8 \\
003 & Sample C & 13.7 & 17.2 \\
004 & Sample D & 15.1 & 14.9 \\
005 & Sample E & 12.9 & 16.5 \\
006 & Sample F & 14.5 & 15.7 \\
007 & Sample G & 13.3 & 17.8 \\
008 & Sample H & 15.8 & 14.2 \\
009 & Sample I & 12.2 & 16.1 \\
010 & Sample J & 14.9 & 15.4 \\
\end{longtable}

% ============================================================================
\section{Advanced Formatting Techniques}
% ============================================================================

\subsection{Line Breaks in Cells}

Use \verb|\makecell| for line breaks within cells:

\begin{table}[h]
\centering
\caption{Line breaks in cells}
\label{tab:linebreaks}
\begin{tabular}{lll}
\toprule
Method & \makecell{Advantages \\ (multiple)} & \makecell{Disadvantages \\ (multiple)} \\
\midrule
A & \makecell[l]{Fast \\ Accurate \\ Simple} & \makecell[l]{Expensive \\ Complex setup} \\
B & \makecell[l]{Cheap \\ Easy} & \makecell[l]{Slow \\ Less accurate} \\
\bottomrule
\end{tabular}
\end{table}

\subsection{Fixed-Width Paragraph Columns}

\begin{table}[h]
\centering
\caption{Using paragraph columns with fixed width}
\label{tab:paragraph}
\begin{tabular}{lp{5cm}r}
\toprule
ID & Description & Value \\
\midrule
A1 & This is a longer text that will automatically wrap within the 5cm wide
column. It demonstrates the p column type. & 100 \\
B2 & Another example with wrapped text showing how paragraph columns handle
longer content automatically. & 200 \\
C3 & Short text. & 150 \\
\bottomrule
\end{tabular}
\end{table}

\subsection{Diagonal Headers}

Using \texttt{diagbox} package (example code):

\begin{verbatim}
\usepackage{diagbox}

\begin{tabular}{|l|c|c|c|}
\hline
\diagbox{Row}{Column} & A & B & C \\
\hline
1 & 10 & 20 & 30 \\
2 & 40 & 50 & 60 \\
\hline
\end{tabular}
\end{verbatim}

% ============================================================================
\section{Complete Example: Research Data}
% ============================================================================

\begin{table}[htbp]
\centering
\caption[Experimental results summary]{Experimental results summary showing
performance metrics across different algorithms and datasets. Bold values
indicate best performance per metric.}
\label{tab:complete}
\rowcolors{2}{gray!5}{white}
\begin{tabular}{
    l
    l
    S[table-format=2.2]
    S[table-format=2.2]
    S[table-format=3.1]
    S[table-format=1.2]
}
\toprule
\multirow{2}{*}{Dataset} & \multirow{2}{*}{Algorithm} &
{Accuracy} & {Precision} & {Time} & {F1} \\
& & {(\%)} & {(\%)} & {(ms)} & {Score} \\
\midrule
\multirow{3}{*}{MNIST} & CNN & 98.45 & 98.32 & 125.5 & 0.98 \\
                       & ResNet & \textbf{99.12} & \textbf{99.05} & 187.3 & \textbf{0.99} \\
                       & VGG & 98.87 & 98.76 & \textbf{110.2} & 0.99 \\
\midrule
\multirow{3}{*}{CIFAR} & CNN & 85.23 & 84.95 & 142.8 & 0.85 \\
                       & ResNet & \textbf{92.56} & \textbf{92.34} & 210.5 & \textbf{0.93} \\
                       & VGG & 89.45 & 89.12 & \textbf{135.7} & 0.89 \\
\midrule
\multirow{3}{*}{ImageNet} & CNN & 72.34 & 71.89 & 1250.0 & 0.72 \\
                          & ResNet & \textbf{85.67} & \textbf{85.23} & 1875.5 & \textbf{0.86} \\
                          & VGG & 78.92 & 78.45 & \textbf{1120.3} & 0.79 \\
\bottomrule
\end{tabular}
\end{table}

Table~\ref{tab:complete} presents comprehensive experimental results across
three datasets and three algorithms. ResNet consistently achieves the highest
accuracy and F1 scores, while VGG offers the best computational efficiency.

% ============================================================================
\section{Table Best Practices}
% ============================================================================

\begin{enumerate}
    \item \textbf{Use \texttt{booktabs}:} Avoid vertical lines; use
    \verb|\toprule|, \verb|\midrule|, \verb|\bottomrule|

    \item \textbf{Caption placement:} Place captions \emph{above} tables
    (opposite to figures)

    \item \textbf{Label after caption:} Always place \verb|\label| after
    \verb|\caption|

    \item \textbf{Alignment:} Align text left, numbers right or on decimal

    \item \textbf{Units in headers:} Include units in column headers, not
    in every cell

    \item \textbf{Consistent precision:} Use the same number of decimal places
    within a column

    \item \textbf{Highlight key results:} Use bold or color sparingly to
    emphasize important values

    \item \textbf{Short captions:} Provide a short caption for the list of
    tables:
    \verb|\caption[short]{long caption}|

    \item \textbf{Avoid overfull tables:} Break wide tables into multiple
    smaller tables if possible

    \item \textbf{Table notes:} Add notes below the table for explanations,
    not in the caption
\end{enumerate}

% ============================================================================
\section{Conclusion}
% ============================================================================

This document has demonstrated comprehensive table creation techniques:

\begin{itemize}
    \item Basic \texttt{tabular} environment with various column types
    \item Professional formatting with \texttt{booktabs}
    \item Multi-row and multi-column cells with \texttt{multirow}
    \item Colored rows and cells with \texttt{xcolor}
    \item Numerical alignment with \texttt{siunitx}
    \item Wide tables with \texttt{tabularx}
    \item Long tables with \texttt{longtable}
    \item Advanced formatting techniques
\end{itemize}

Key packages for professional tables:
\begin{verbatim}
\usepackage{booktabs}    % Professional rules
\usepackage{multirow}    % Multi-row cells
\usepackage{siunitx}     % Number alignment
\usepackage[table]{xcolor} % Colored rows
\usepackage{tabularx}    % Auto-width columns
\usepackage{longtable}   % Multi-page tables
\end{verbatim}

\end{document}

% ============================================================================
% COMPILATION NOTES
% ============================================================================
% Compile with: pdflatex tables_demo.tex (run twice for cross-references)
%
% Common issues:
% 1. \multirow requires the multirow package
% 2. S columns (siunitx) require headers in braces: {Header}
% 3. Vertical lines often look unprofessional - use booktabs instead
% 4. longtable cannot be used inside a table environment
% 5. Remember to \centering before tabular in a table environment
% ============================================================================
