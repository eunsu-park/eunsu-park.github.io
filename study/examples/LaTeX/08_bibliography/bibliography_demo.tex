% bibliography_demo.tex
% Demonstrates BibLaTeX citation management with various citation styles
% Compile: pdflatex -> biber -> pdflatex -> pdflatex

\documentclass[12pt,a4paper]{article}

% BibLaTeX package for bibliography management
\usepackage[
    backend=biber,
    style=authoryear,  % Options: numeric, authoryear, alphabetic, apa
    sorting=nyt,       % Sort by name, year, title
    maxbibnames=99,
    maxcitenames=2,
    giveninits=true
]{biblatex}

% Add the bibliography file
\addbibresource{references.bib}

\usepackage[utf8]{inputenc}
\usepackage[T1]{fontenc}
\usepackage{hyperref}
\usepackage{csquotes}

\title{Bibliography Management with BibLaTeX}
\author{LaTeX Student}
\date{\today}

\begin{document}

\maketitle

\begin{abstract}
This document demonstrates various citation styles and bibliography management
using BibLaTeX. We explore different citation commands, reference types, and
formatting options commonly used in academic writing.
\end{abstract}

\tableofcontents

\section{Introduction}

Proper citation is essential in academic writing. BibLaTeX provides a modern,
flexible system for managing bibliographies in LaTeX documents
\parencite{lamport1994latex}. This example demonstrates various citation
techniques.

\section{Different Citation Styles}

\subsection{Parenthetical Citations}

The most common form of citation uses parentheses. For example, deep learning
has revolutionized computer vision \parencite{goodfellow2016deep}. When citing
multiple works, you can use \parencite{lecun2015deep, schmidhuber2015deep}.

\subsection{Textual Citations}

When the author is part of the sentence, use textual citations. As
\textcite{hinton2006fast} demonstrated, deep belief networks can learn
hierarchical representations. Similarly, \textcite{krizhevsky2012imagenet}
showed breakthrough results on ImageNet classification.

\subsection{Author-Only and Year-Only}

Sometimes you need just the author: \citeauthor{vaswani2017attention} introduced
the Transformer architecture. Or just the year: This was published in
\citeyear{vaswani2017attention}.

\section{Reference Types}

\subsection{Journal Articles}

Recurrent neural networks have shown impressive results in sequence modeling
\parencite{hochreiter1997long}. The LSTM architecture addressed the vanishing
gradient problem \parencite{bengio1994learning}.

\subsection{Books and Conference Papers}

For comprehensive introductions to machine learning, refer to
\textcite{bishop2006pattern}. The ResNet architecture \parencite{he2016deep}
introduced skip connections that enabled training of very deep networks.

\subsection{Online Resources}

Many important papers are now published on arXiv \parencite{arxiv2020gpt3}.
The TensorFlow library documentation \parencite{tensorflow2015whitepaper}
provides practical implementation guidance.

\section{Advanced Citation Features}

\subsection{Multiple Citations}

Recent work in transformer models \parencites{vaswani2017attention}%
{devlin2019bert}{brown2020language} has transformed natural language processing.

\subsection{Prenotes and Postnotes}

For specific page references, use postnotes: \parencite[p.~42]{goodfellow2016deep}.
You can also add prenotes: \parencite[see][ch.~6]{bishop2006pattern}.

\subsection{Suppressing Author or Year}

In some cases, you want to suppress the author \parencite*{lamport1994latex}
or cite without parentheses \cite{lecun2015deep}.

\section{Full Citations in Text}

Sometimes you need a full citation in the text rather than at the end:
\fullcite{hinton2006fast}

\section{Nocite Examples}

You can include references in the bibliography without citing them in text
using \verb|\nocite|. This is useful for "recommended reading" sections.

% Uncomment to include uncited reference
% \nocite{schmidhuber2015deep}

\section{Conclusion}

BibLaTeX offers powerful and flexible bibliography management. The
\texttt{biber} backend provides Unicode support, flexible sorting, and
advanced features that surpass traditional BibTeX \parencite{lamport1994latex}.

\subsection{Compilation Instructions}

To compile this document:
\begin{enumerate}
    \item Run: \texttt{pdflatex bibliography\_demo}
    \item Run: \texttt{biber bibliography\_demo}
    \item Run: \texttt{pdflatex bibliography\_demo}
    \item Run: \texttt{pdflatex bibliography\_demo}
\end{enumerate}

% Print the bibliography
\printbibliography[title={References}]

\end{document}
